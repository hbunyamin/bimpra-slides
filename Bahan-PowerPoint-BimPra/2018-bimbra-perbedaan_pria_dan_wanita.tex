\documentclass{beamer}

%\usepackage{lmodern}
\usepackage[font=scriptsize,skip=0pt,justification=justified,singlelinecheck=false]{caption}
%\usepackage{enumitem}
\usepackage{natbib}
\usepackage{bm}
\usepackage{mathtools}
\usepackage[makeroom]{cancel}

% Using underline
\usepackage{soul}


%remove the icon
\setbeamertemplate{bibliography item}{}

%remove line breaks
\setbeamertemplate{bibliography entry title}{}
\setbeamertemplate{bibliography entry location}{}
\setbeamertemplate{bibliography entry note}{}
% Use number for caption
\setbeamertemplate{caption}[numbered]{}

\newtheorem{mydef}[theorem]{\Large \underline{\textbf{Definisi}}}

\makeatletter
\def\th@mystyle{%
    \normalfont % body font
    \setbeamercolor{block title example}{bg=blue,fg=white}
    \setbeamercolor{block body example}{bg=blue!20,fg=black}
    \def\inserttheoremblockenv{exampleblock}
  }
\makeatother
\theoremstyle{mystyle}
\newtheorem*{remark}{\textbf{Definition}}


% This file is a solution template for:

% - Talk at a conference/colloquium.
% - Talk length is about 20min.
% - Style is ornate.

% Copyright 2004 by Till Tantau <tantau@users.sourceforge.net>.
%
% In principle, this file can be redistributed and/or modified under
% the terms of the GNU Public License, version 2.
%
% However, this file is supposed to be a template to be modified
% for your own needs. For this reason, if you use this file as a
% template and not specifically distribute it as part of a another
% package/program, I grant the extra permission to freely copy and
% modify this file as you see fit and even to delete this copyright
% notice.  


\mode<presentation>
{
%  \usetheme{AnnArbor} % 
%	\usetheme{Frankfurt}
   \usetheme{Madrid}
%	\usetheme{Darmstadt}
  % or ...

%  \setbeamercovered{transparent}
  % or whatever (possibly just delete it)
}


\usepackage[english]{babel}
% or whatever

\usepackage[latin1]{inputenc}
% or whatever

\usepackage{times}
\usepackage[T1]{fontenc}
\usepackage{wasysym}

% Define absolute and norm
\DeclarePairedDelimiter\abs{\lvert}{\rvert}%
\DeclarePairedDelimiter\norm{\lVert}{\rVert}%

% ==============================
%       Redefine emphasize
% ==============================
\let\emph\relax % there's no \RedeclareTextFontCommand
\DeclareTextFontCommand{\emph}{\bfseries\em}


% Swap the definition of \abs* and \norm*, so that \abs
% and \norm resizes the size of the brackets, and the 
% starred version does not.
\makeatletter
\let\oldabs\abs
\def\abs{\@ifstar{\oldabs}{\oldabs*}}
%
\let\oldnorm\norm
\def\norm{\@ifstar{\oldnorm}{\oldnorm*}}
\makeatother

\usepackage{color}
\definecolor{myblue}{rgb}{.8,.8,1}
\usepackage{empheq}
% Or whatever. Note that the encoding and the font should match. If T1
% does not look nice, try deleting the line with the fontenc.

\newlength\mytemplen
\newsavebox\mytempbox

\makeatletter
\newcommand\mybluebox{%
    \@ifnextchar[%]
       {\@mybluebox}%
       {\@mybluebox[0pt]}}

\def\@mybluebox[#1]{%
    \@ifnextchar[%]
       {\@@mybluebox[#1]}%
       {\@@mybluebox[#1][0pt]}}

\def\@@mybluebox[#1][#2]#3{
    \sbox\mytempbox{#3}%
    \mytemplen\ht\mytempbox
    \advance\mytemplen #1\relax
    \ht\mytempbox\mytemplen
    \mytemplen\dp\mytempbox
    \advance\mytemplen #2\relax
    \dp\mytempbox\mytemplen
    \colorbox{myblue}{\hspace{1em}\usebox{\mytempbox}\hspace{1em}}}

\makeatother

\title[Perbedaan Laki-Laki dan Perempuan] % (optional, use only with long paper titles)
{\textbf{Perbedaan Laki-Laki dan Perempuan}}

%\subtitle
%{\textit{Perbedaan Pria dan Wanita}}

\author[Hendra Bunyamin] % (optional, use only with lots of authors)
{Hendra Bunyamin}
%{F.~Author\inst{1} \and S.~Another\inst{2}} --> original
% - Give the names in the same order as the appear in the paper.
% - Use the \inst{?} command only if the authors have different
%   affiliation.

\institute[ ] % (optional, but mostly needed)
{
%  \inst{1}%
  \hfill \break
  \hfill \break
  \hfill \break
  \large
  Bimbingan Pranikah\\
  GKI Anugerah
%  \and
%  \inst{2}%
%  Department of Theoretical Philosophy\\
%  University of Elsewhere
}
% - Use the \inst command only if there are several affiliations.
% - Keep it simple, no one is interested in your street address.

%\date[CFP 2003] % (optional, should be abbreviation of conference name)
%{Conference on Fabulous Presentations, 2003}
% - Either use conference name or its abbreviation.
% - Not really informative to the audience, more for people (including
%   yourself) who are reading the slides online

\subject{PowerPoint}
% This is only inserted into the PDF information catalog. Can be left
% out. 

% If you have a file called "university-logo-filename.xxx", where xxx
% is a graphic format that can be processed by latex or pdflatex,
% resp., then you can add a logo as follows:

\pgfdeclareimage[height=1.5cm]{university-logo}{logo-gkia-komit}
\logo{\pgfuseimage{university-logo}}


% Delete this, if you do not want the table of contents to pop up at
% the beginning of each subsection:
\AtBeginSection[]
{
  \begin{frame}<beamer>{\textbf{Outline}}
    \tableofcontents[currentsection,currentsection]
  \end{frame}
}


% If you wish to uncover everything in a step-wise fashion, uncomment
% the following command: 

%\beamerdefaultoverlayspecification{<+->}

\begin{document}

\begin{frame}
  \titlepage
\end{frame}

\begin{frame}{\textbf{Outline}}
  \tableofcontents
  % You might wish to add the option [pausesections]
\end{frame}


% Structuring a talk is a difficult task and the following structure
% may not be suitable. Here are some rules that apply for this
% solution: 

% - Exactly two or three sections (other than the summary).
% - At *most* three subsections per section.
% - Talk about 30s to 2min per frame. So there should be between about
%   15 and 30 frames, all told.

% - A conference audience is likely to know very little of what you
%   are going to talk about. So *simplify*!
% - In a 20min talk, getting the main ideas across is hard
%   enough. Leave out details, even if it means being less precise than
%   you think necessary.
% - If you omit details that are vital to the proof/implementation,
%   just say so once. Everybody will be happy with that.

%\begin{frame}{Make Titles Informative. Use Uppercase Letters.}{Subtitles are optional.}

\section{Kejadian 1: Kesetaraan}
\begin{frame}{\textbf{Kejadian 1: Kesetaraan}}
	\begin{itemize}
		\item<2-> \textbf{Kejadian 1:26-27}:\\
		Berfirmanlah Allah: "Baiklah Kita menjadikan manusia menurut gambar dan rupa Kita, supaya mereka berkuasa atas ikan-ikan di laut dan burung-burung di udara dan atas ternak dan atas seluruh bumi dan atas segala binatang melata yang merayap di bumi." Maka Allah menciptakan manusia itu menurut gambar-Nya, menurut gambar Allah diciptakan-Nya dia; \emph{laki-laki dan perempuan} diciptakan-Nya mereka. 
		\hfill \break
		\item<3-> Laki-laki dan perempuan \emph{setara}.
		\hfill \break
		\item<4-> \textbf{Kejadian 1} berbicara mengenai \textbf{kesetaraan}~\citep{pritchard1992menwomen}. \\
		Marilah kita lihat apa yang tertulis di \textbf{Kejadian 2}.
	\end{itemize}
\end{frame}

\section{Kejadian 2: Perbedaan Peran}
\begin{frame}{\textbf{Kejadian 2: Perbedaan Peran}}
	\begin{itemize}
		\item<2-> Adam diciptakan pertama (Kej. 2:7).		
		\item<3-> Adam diberi mandat untuk mengusahakan dan memelihara taman (Kej. 2:15).
		\item<4-> Adam diberi peringatan mengenai pohon pengetahuan tentang yang baik dan yang jahat (Kej. 2:16-17).
		\item<5-> "\ldots Aku akan menjadikan \textit{penolong} baginya, yang sepadan dengan dia." (Kej. 2:18, Ef. 5:22).
		\item<6-> Adam memberi nama semua binatang (Kej. 2:19-20).
		\item<7-> Hawa dibentuk dari tulang rusuk Adam (Kej. 2:21-22).
		\item<8-> Adam memberi nama \textit{perempuan} (Kej. 2:23).
		\item<9-> Seorang laki-laki akan meninggalkan ayahnya dan ibunya dan bersatu dengan isterinya (Kej. 2:24).
	\end{itemize}
\end{frame}

\section{Kejadian 3: Ketika Dosa Masuk ...}
\begin{frame}{\textbf{Kejadian 3: Ketika Dosa Masuk ...}}
	Kejadian 3 memperlihatkan bagaimana dosa masuk ke dunia.
	\begin{itemize}
		\item<2-> Ular mendekati Hawa --- bukan Adam (membalikkan urutan Tuhan). (3:1)
		\item<3-> Hawa berdosa pertama (3:6; 1 Tim 2:13-14). 
		\item<4-> Adam berdosa (3:6; 1 Tim 2:13-14). 
		\item<5-> Tuhan berkata kepada Adam yang adalah kepala taman. (3:9)
		\item<6-> Tuhan menghukum Adam dan Hawa berbeda. (3:16-19)
	\end{itemize}
	
	\only<7->{\textit{Siapakah yang berdosa lebih besar?}\\} 
	\only<8->{\textbf{Adam}, karena dia memperoleh mandat, dia tahu ini salah tapi dia tetap melakukan.}

	\bigskip

	\only<9->{\textit{Siapakah yang bertanggung jawab?}\\} 	
	\only<10->{\textbf{Adam}, karena dia adalah kepala istrinya, kepala taman, kepala representasi umat manusia. (Rom 5:12)}	
\end{frame}

\begin{frame}{\textbf{Kejadian 3: Akibat Dosa untuk Hawa}}
	\textbf{Kejadian 3:16} \\
	\textit{Firman-Nya kepada perempuan itu: "Susah payahmu waktu mengandung akan Kubuat sangat banyak; dengan kesakitan engkau akan melahirkan anakmu; namun engkau akan berahi kepada suamimu dan ia akan berkuasa atasmu.} 
	
	\bigskip
	
	Menurut versi NLT, "\textit{And though you may desire to control your husband, ...}"	
	
	\bigskip
	
	Hawa dihukum di 2 area yang dekat dengan hatinya:
	\begin{itemize}
		\item childbirth $\Rightarrow$ physical attachment;
		\item her relationship to Adam $\Rightarrow$ personal attachment $\Rightarrow$ heartache and battle.
	\end{itemize}	 
\end{frame}


\begin{frame}{\textbf{Kejadian 3: Akibat Dosa untuk Adam (1/2)}}
	\textbf{Kejadian 3:17-19} \\
	\textit{Lalu firman-Nya kepada manusia itu: "Karena engkau mendengarkan perkataan isterimu dan memakan dari buah pohon, yang telah Kuperintahkan kepadamu: Jangan makan dari padanya, maka terkutuklah tanah karena engkau; dengan bersusah payah engkau akan mencari rezekimu dari tanah seumur hidupmu: semak duri dan rumput duri yang akan dihasilkannya bagimu, dan tumbuh-tumbuhan di padang akan menjadi makananmu; dengan berpeluh engkau akan mencari makananmu, sampai engkau kembali lagi menjadi tanah, karena dari situlah engkau diambil; sebab engkau debu dan engkau akan kembali menjadi debu.}
\end{frame}

\begin{frame}{\textbf{Kejadian 3: Akibat Dosa untuk Adam (2/2)}}
	Adam dihukum berbeda dengan Hawa, yaitu
	\begin{itemize}
		\item Ground cursed,
		\item Painful toil,
		\item Thorns and thistles,
		\item Sweat of your brow.
	\end{itemize}
	Penekanannya berbeda sekali. Hawa dihukum di area relationships dan Adam dihukum di area pekerjaannya di dunia.
	
	\bigskip
	
	\textit{Adam lost the sense of completion that comes from powerfully subduing his world and meaningfully touching a woman who would, as a priority, value his work and enjoy his involvement}~\citep{crabb2013menandwomen}.

	\bigskip
	
	Akibat dosa, sebelumnya hubungan pria dan wanita \textit{cooperation} menjadi \textit{competition}.

\end{frame}

\section{Realitas Abad 21}
\begin{frame}{\textbf{Realitas Abad 21}}
	Dua ribu tahun sudah lewat sejak kejadian di Taman Eden, \textit{competition is still the name of the game}.
	
	\begin{enumerate}
		\item<2-> \textit{Male Chauvinism}: "intentional disrespect for women".
			\begin{itemize}
				\item<3-> \textit{Sexual harrassment} --- Harvey Weinstein di Hollywood.
				\item<4-> \textit{Demeaning Comments and Jokes}.
				\item<5-> \textit{Physical/Emotional/Sexual Abuse} --- 25\% of the women in America have been sexually or physically abused.
				\item<6-> \textit{Marital Unfaithfulness}.				
			\end{itemize}
		\item<2-> \textit{Radical Feminism}: "the secular glorification of women" $\Rightarrow$ take control of their own destinies.		
			\begin{itemize}
				\item<7-> \textit{Denial of the Differences} --- "the only difference is biology". 
				\item<8-> \textit{Demeaning of Motherhood} --- Motherhood, Homemaking, Childrearing = "second class" professions.
				\item<9-> \textit{Glorification of Immorality in the Name of Liberation} --- Kebebasan seksual $\Rightarrow$ tidur di mana saja, dengan siapa saja, kapan saja, dan kalau punya baby tinggal aborsi.
			\end{itemize}
	\end{enumerate}
\end{frame}

\section{Perbedaan Pria \& Wanita secara Biblikal}
\begin{frame}{\textbf{Secara Biblical ...}}
	\begin{itemize}
		\item<2-> To Be a Man Means to Be ...
		\begin{itemize}
			\item<2-> A Protector 
			\item<2-> A Provider
			\item<2-> A Self-Sacrificing Leader
		\end{itemize}
		
		\bigskip		
		
		\item<3-> To Be a Woman Means to Be ...
		\begin{itemize}
			\item<3-> A Relational Encourager
			\item<3-> An Intuitive Value-Giver
			\item<3-> A Self-Sacrificing Completer
		\end{itemize}
	\end{itemize}

\end{frame}



\section{8 Perbedaan Terbesar antara Saudara dengan Pasangan Saudara}
\begin{frame}{\textbf{8 Perbedaan antara Lelaki dan Perempuan}}
	\begin{columns}[c]		
		\column{1.5in}
		\begin{itemize}
			\item<2-> Fisik
			\hfill \break
			\item<3-> Uang
			\hfill \break
			\item<4-> Relasi
			\hfill \break
			\item<5-> Verbal
		\end{itemize}		
		\column{2.5in}
		\begin{itemize}
			\item<6-> Kebutuhan 
			\hfill \break
			\item<7-> Motivasi
			\hfill \break
			\item<8-> Aktivitas
			\hfill \break
			\item<9-> Kasih Sayang
		\end{itemize}				
	\end{columns}
\end{frame}

\begin{frame}{\textbf{Perbedaan mengenai Fisik (\textit{Physical})}}
	\begin{itemize}
		\item<2-> Koneksi otak laki-laki dan perempuan yang berbeda.
		\hfill \break
		\item<3-> Laki-laki lebih dominan otak sebelah kiri (logika) dan perempuan lebih seimbang\footnote{https://www.allprodad.com/the-10-biggest-differences-between-you-and-your-wife}.
		\hfill \break
		\item<4-> Akibatnya, laki-laki cenderung fokus pada satu pekerjaan sampai selesai.
		\hfill \break		
		\item<5-> Sedangkan, perempuan dapat mengerjakan \textit{multiple tasks at a time, sometimes completing them or leaving them for a later}.\\
		\hfill \break
	\end{itemize}
\end{frame}

\begin{frame}{\textbf{Perbedaan mengenai Pengeluaran (\textit{Money})}}
	\begin{itemize}
		\item Laki-laki mengeluarkan uang untuk sesuatu yang mereka butuhkan.
		\hfill \break
		\item<2-> Umumnya perempuan akan membeli sesuatu yang mereka \textit{tidak} butuhkan jika ada \textit{sale}.
	\end{itemize}
\end{frame}

\begin{frame}{\textbf{Perbedaan mengenai Relasi (\textit{Relationally})}}
	\begin{itemize}
		\item<2-> Perempuan membangun hubungan dengan \textit{sharing emotions} sedangkan laki-laki \textit{sharing activities --- doing things together}.
		\hfill \break
		\item<3-> Menurut \citet{mayemotional2017}, perempuan \textit{lebih ekspresif dalam menampilkan emosi} dilihat dari raut muka.\\
		Ketika ada iklan lucu, lebih banyak perempuan yang senyum daripada laki-laki dan senyum perempuan lebih lama daripada laki-laki.
	\end{itemize}
\end{frame}

\begin{frame}{\textbf{Perbedaan tentang Pengungkapan Lisan (\textit{Verbally})}}
	\begin{itemize}
		\item<1-> Laki-laki $\Longrightarrow$ padat (\textit{concise}) dan bahasa pendek.
		\hfill \break
		\item<2-> Perempuan $\Longrightarrow$ panjang dan detil.
		\hfill \break
		\item<3-> \textit{Story} tentang Rachel dan Ross dari TV series "Friends".
	\end{itemize}
\end{frame}

\begin{frame}{\textbf{Perbedaan mengenai Kebutuhan (\textit{Needs})}}
		\begin{itemize}
			\item Perempuan memiliki kebutuhan untuk \textit{didengar}.
			\hfill \break
			\item<2-> Hal ini dapat konflik dengan laki-laki yang lebih suka \textit{memberikan solusi} daripada \textit{mendengar}.
		\end{itemize}			
\end{frame}

%\begin{frame}{\textbf{Perbedaan mengenai Nilai (\textit{Values})}}
%	\begin{itemize}
%		\item<1-> Perempuan mungkin $\Rightarrow$ kerapihan (\textit{neatness}) dan keteraturan (\textit{order}).
%		\hfill \break
%		\item<2-> Laki-laki mungkin $\Rightarrow$ \textit{power} dan efisiensi.
%		\hfill \break
%		\item<3-> \textbf{Solusi}: idealnya, semakin lama menikah, nilai akan semakin menyatu.
%	\end{itemize}
%\end{frame}

\begin{frame}{\textbf{Perbedaan mengenai Motivasi (\textit{Motivationally})}}
	\citet{robbins-basic-needs-14} berkata bahwa terdapat kebutuhan-kebutuhan (\textit{needs}) yang memotivasi manusia.
	\begin{itemize}
		\item<2-> \textit{Need 1: Certainty/Comfort}.
		\hfill \break
		\item<3-> \textit{Need 3: Significance}.
		\hfill \break
		\item<4-> \textit{Need 4: Love \& Connection}
		\hfill \break
		\item<5-> \textit{Need 5: Growth}
		\hfill \break
		\item<6-> \textit{Need 6: Contribution}		
	\end{itemize}
\end{frame}

\begin{frame}{\textbf{Perbedaan mengenai Aktivitas (\textit{Activities})}\footnote{https://www.allprodad.com/10-creative-ways-spend-time-wife}}
	\begin{itemize}
		\item<2-> \textit{Exercise together}.
		\hfill \break
		\item<3-> \textit{Sign up for a class together}.
		\hfill \break
		\item<4-> \textit{Volunteer together}.
		\hfill \break
		\item<5-> \textit{Take it in turns to organize a "mystery adventure date"}.
		\hfill \break
		\item<6-> \textit{Step out together down memory lane}.
	\end{itemize}
\end{frame}

\begin{frame}{\textbf{Perbedaan mengenai \textit{Affection} (1/3)}}
	\citet{chapman2015love} menjelaskan bahwa terdapat 5 bahasa kasih untuk mempertahankan cinta yang \textit{membara}.
	\begin{itemize}
		\item<2-> \emph{Words of Affirmation}. \\
		"\textit{My husband's encouraging words give me confidence}",\\
		"\textit{It makes me feel really good when my husband tells me he appreciates me}",\\
		"\textit{I love hearing my husband tell me that he missed me}".
		\hfill \break
		\item<3-> \emph{Quality Time}. \\
		"\textit{My husband doesn't interrupt me when I am talking, and I like that}", \\
		"\textit{It doesn't matter where we go, I just like going places with my husband}", \\
		"\textit{I love to watch movies with my husband}".
	\end{itemize}
\end{frame}

\begin{frame}{\textbf{Perbedaan mengenai \textit{Affection} (2/3)}}
	\citet{chapman2015love} menjelaskan bahwa terdapat 5 bahasa kasih untuk mempertahankan cinta yang \textit{membara}.
	\begin{itemize}
		\item<2-> \emph{Receiving Gifts}. \\
		"\textit{My husband lets me know he loves me by giving me gifts}", \\
		"\textit{I never get tired of receiving gifts from my husband}", \\
		"\textit{I love surprise gifts from my husband}".
		\hfill \break
		\item<3-> \emph{Acts of Service}. \\
		"\textit{My husband shows his love by helping me without me having to ask}",\\
		"\textit{My husband is good about asking how he can help when I'm tired}", \\
		"\textit{It means a lot to me when my husband helps me despite being busy}".		
	\end{itemize}
\end{frame}

\begin{frame}{\textbf{Perbedaan mengenai \textit{Affection} (3/3)}}
	\begin{itemize}
		\item<2-> \emph{Physical Touch}. \\
		"\textit{I love cuddling with my husband}", \\
		"\textit{I love it that my husband can't keep his hands off me}", \\
		"\textit{I love hugging and kissing my husband after we've been apart for a while}". \\			
	\end{itemize}
\end{frame}


\section{Difference Between Man and Woman by Dr. Chirag N. Patel}
\begin{frame}{\textbf{Video: Difference Between Man and Woman}}
	Dr. Chirag N. Patel menjelaskan perbedaan laki-laki dan perempuan dari segi 
	\begin{itemize}
		\item pakaian (\textit{dress}), 
		\item \textit{sharing a bed}, 
		\item fokus, 
		\item cara berpikir, dan
		\item berbagai macam kebiasaan.
	\end{itemize}			
\end{frame}

\section{Conclusion: Biblical View}
\begin{frame}{\textbf{Conclusion}}
	\begin{itemize}
		\item Pria sebagai Kepala Keluarga berarti \textit{man has a holy obligation before God to lay down his life for his wife, his children}.
		
		\bigskip		
				
		\item For a husband, it means that he has his wife's best interests at heart, that he sacrifices his own desires for hers, that he puts her first always in his affections. 
		
		\bigskip		
		
		\item For a wife, submission in this context means believing that God is able to work through her husband to accomplish HIS will in her life, to protect her interests, and to meet her deepest needs.
	\end{itemize}
\end{frame}

%\begin{frame}{Blocks}
%\begin{block}{Block Title}
%You can also highlight sections of your presentation in a block, with it's own title
%\end{block}
%\begin{theorem}
%There are separate environments for theorems, examples, definitions and proofs.
%\end{theorem}
%\begin{example}
%Here is an example of an example block.
%\end{example}
%\end{frame}


% All of the following is optional and typically not needed. 
%\appendix
%\section<presentation>*{\appendixname}
%\subsection<presentation>*{For Further Reading}
%
\begin{frame}[allowframebreaks]
  \frametitle<presentation>{\textbf{Referensi}}
    {\footnotesize
    \bibliographystyle{apalike}
    \bibliography{references}
    }    
\end{frame}

%\makeatletter % to change template
%    \setbeamertemplate{headline}[default] % not mandatory, but I though it was better to set it blank
%    \def\beamer@entrycode{\vspace*{-\headheight}} % here is the part we are interested in :)
%\makeatother

%\begin{frame}[plain]
%		\centering\includegraphics[scale=0.5]{Logo-Maranatha-Untuk-Belakang-02}	
%\end{frame}

\begin{frame}[plain]
		\centering\includegraphics[scale=1]{thank-you}	
		hendra.bunyamin@gmail.com
\end{frame}


\end{document}



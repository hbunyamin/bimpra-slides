\documentclass[pdf]{beamer}

%\usepackage{lmodern}
\usepackage[font=scriptsize,skip=0pt,justification=justified,singlelinecheck=false]{caption}
%\usepackage{enumitem}
\usepackage{natbib}
\usepackage{bm}
\usepackage{mathtools}
\usepackage[makeroom]{cancel}
\usepackage{minted}

% Using underline
\usepackage{soul}

\usepackage{algorithm}
\usepackage{algpseudocode}
%\usepackage[ruled,vlined]{algorithm2e}
%\usepackage{clrscode3e}


%\usepackage{adjustbox}

%remove the icon
\setbeamertemplate{bibliography item}{}

%remove line breaks
\setbeamertemplate{bibliography entry title}{}
\setbeamertemplate{bibliography entry location}{}
\setbeamertemplate{bibliography entry note}{}
% Use number for caption
\setbeamertemplate{caption}[numbered]{}

\newtheorem{mydef}[theorem]{\Large \underline{\textbf{Definisi}}}

\makeatletter
\def\th@mystyle{%
    \normalfont % body font
    \setbeamercolor{block title example}{bg=blue,fg=white}
    \setbeamercolor{block body example}{bg=blue!20,fg=black}
    \def\inserttheoremblockenv{exampleblock}
  }
\makeatother
\theoremstyle{mystyle}
\newtheorem*{remark}{\textbf{Definition}}


% This file is a solution template for:

% - Talk at a conference/colloquium.
% - Talk length is about 20min.
% - Style is ornate.

% Copyright 2004 by Till Tantau <tantau@users.sourceforge.net>.
%
% In principle, this file can be redistributed and/or modified under
% the terms of the GNU Public License, version 2.
%
% However, this file is supposed to be a template to be modified
% for your own needs. For this reason, if you use this file as a
% template and not specifically distribute it as part of a another
% package/program, I grant the extra permission to freely copy and
% modify this file as you see fit and even to delete this copyright
% notice.  


\mode<presentation>
{
%  \usetheme{AnnArbor} % 
%	\usetheme{Frankfurt}
   \usetheme{Madrid}
  % or ...

%  \setbeamercovered{transparent}
  % or whatever (possibly just delete it)
}


\usepackage[english]{babel}
% or whatever

\usepackage[latin1]{inputenc}
% or whatever

\usepackage{times}
\usepackage[T1]{fontenc}
\usepackage{wasysym}

% Define absolute and norm
\DeclarePairedDelimiter\abs{\lvert}{\rvert}%
\DeclarePairedDelimiter\norm{\lVert}{\rVert}%

% ==============================
%       Redefine emphasize
% ==============================
\let\emph\relax % there's no \RedeclareTextFontCommand
\DeclareTextFontCommand{\emph}{\bfseries\em}


% Swap the definition of \abs* and \norm*, so that \abs
% and \norm resizes the size of the brackets, and the 
% starred version does not.
\makeatletter
\let\oldabs\abs
\def\abs{\@ifstar{\oldabs}{\oldabs*}}
%
\let\oldnorm\norm
\def\norm{\@ifstar{\oldnorm}{\oldnorm*}}
\makeatother

\usepackage{color}
\definecolor{myblue}{rgb}{.8,.8,1}
\usepackage{empheq}
% Or whatever. Note that the encoding and the font should match. If T1
% does not look nice, try deleting the line with the fontenc.

\newlength\mytemplen
\newsavebox\mytempbox

\makeatletter
\newcommand\mybluebox{%
    \@ifnextchar[%]
       {\@mybluebox}%
       {\@mybluebox[0pt]}}

\def\@mybluebox[#1]{%
    \@ifnextchar[%]
       {\@@mybluebox[#1]}%
       {\@@mybluebox[#1][0pt]}}

\def\@@mybluebox[#1][#2]#3{
    \sbox\mytempbox{#3}%
    \mytemplen\ht\mytempbox
    \advance\mytemplen #1\relax
    \ht\mytempbox\mytemplen
    \mytemplen\dp\mytempbox
    \advance\mytemplen #2\relax
    \dp\mytempbox\mytemplen
    \colorbox{myblue}{\hspace{1em}\usebox{\mytempbox}\hspace{1em}}}

\makeatother

\title[Time Complexity \& Big O Notation] % (optional, use only with long paper titles)
{\textbf{Matematika Diskrit}}

\subtitle
{Algorithm Analysis \& Big-O Notation}

\author[Hendra Bunyamin] % (optional, use only with lots of authors)
{Hendra Bunyamin}
%{F.~Author\inst{1} \and S.~Another\inst{2}} --> original
% - Give the names in the same order as the appear in the paper.
% - Use the \inst{?} command only if the authors have different
%   affiliation.

\institute[ ] % (optional, but mostly needed)
{
%  \inst{1}%
  Jurusan Teknik Informatika\\
  Fakultas Teknologi Informasi\\
  Universitas Kristen Maranatha
%  \and
%  \inst{2}%
%  Department of Theoretical Philosophy\\
%  University of Elsewhere
}
% - Use the \inst command only if there are several affiliations.
% - Keep it simple, no one is interested in your street address.

%\date[CFP 2003] % (optional, should be abbreviation of conference name)
%{Conference on Fabulous Presentations, 2003}
% - Either use conference name or its abbreviation.
% - Not really informative to the audience, more for people (including
%   yourself) who are reading the slides online

\subject{Discrete Mathematics}
% This is only inserted into the PDF information catalog. Can be left
% out. 

% If you have a file called "university-logo-filename.xxx", where xxx
% is a graphic format that can be processed by latex or pdflatex,
% resp., then you can add a logo as follows:

\pgfdeclareimage[height=1.5cm]{university-logo}{logo-mcu}
\logo{\pgfuseimage{university-logo}}


% Delete this, if you do not want the table of contents to pop up at
% the beginning of each subsection:
\AtBeginSection[]
{
  \begin{frame}<beamer>{\textbf{Outline}}
    \tableofcontents[currentsection,currentsection]
  \end{frame}
}


% If you wish to uncover everything in a step-wise fashion, uncomment
% the following command: 

%\beamerdefaultoverlayspecification{<+->}

\begin{document}

\begin{frame}
  \titlepage
\end{frame}

\begin{frame}{\textbf{Outline}}
  \tableofcontents
  % You might wish to add the option [pausesections]
\end{frame}


% Structuring a talk is a difficult task and the following structure
% may not be suitable. Here are some rules that apply for this
% solution: 

% - Exactly two or three sections (other than the summary).
% - At *most* three subsections per section.
% - Talk about 30s to 2min per frame. So there should be between about
%   15 and 30 frames, all told.

% - A conference audience is likely to know very little of what you
%   are going to talk about. So *simplify*!
% - In a 20min talk, getting the main ideas across is hard
%   enough. Leave out details, even if it means being less precise than
%   you think necessary.
% - If you omit details that are vital to the proof/implementation,
%   just say so once. Everybody will be happy with that.

%\begin{frame}{Make Titles Informative. Use Uppercase Letters.}{Subtitles are optional.}

\section{How to Measure the Performance of an Algorithm}
\begin{frame}{How to measure the performance of an algorithm? (1/2)}
	\begin{itemize}
		\item How do you do \emph{measure} your code? 
		\item Would you clock \textbf{how long} it takes to run?
		\item What if you are running the same program on a mobile device or a quantum computer?\footnote{https://adrianmejia.com/categories/programming/data-structures-and-algorithms-dsa}
	\end{itemize}
	\begin{center}
	\noindent\fbox{\textbf{Essentially, what is the running time of an algorithm?}}	
	\end{center}
\end{frame}

%\noindent\fbox{%
%    \parbox{\textwidth}{%
%        The quick brown fox jumps right over the lazy dog. 
%    }%
%}

\begin{frame}{How to measure the performance of an algorithm? (2/2)}
	Measuring a "running time"~\citep{roughgarden2017illuminatedpart1} $\Longrightarrow$
	\begin{itemize}
		\item count the number of lines of code executed in a concrete implementation of the algorithm or
		\item compute the number of times the \emph{basic operation} is executed.
	\end{itemize}
	
	\bigskip	
	
	\emph{Basic operations} or "\textbf{primitive}" operations~\citep{levitin2012introtothedesign} are addition, subtraction, multiplication, division, comparison, and copying.
\end{frame}

\begin{frame}{Example 1: How Many Additions and Multiplications (1/2)}
	\begin{algorithm}[H]
		\caption{AdditionAndMultiplication1}
		\label{algo:addition-and-multiplication-1}
		\begin{algorithmic}[1] 
			\Function{AdditionAndMultiplication1}{$n$}
			\State $p=0$
			\State $x=2$
			\For{$i = 2$ to $n$}
				\State $p = (p+i) \cdot x$
			\EndFor
			\EndFunction
		\end{algorithmic}
	\end{algorithm}			
	\bigskip	
		
	How many additions, copying, and multiplications?
\end{frame}

\begin{frame}{Example 1: How Many Additions and Multiplications (2/2)}
	Number of operations = $3n + 2$
\end{frame}

\begin{frame}{Example 2: How Many Additions and Multiplications (1/2)}
		\begin{algorithm}[H]
		\caption{AdditionAndMultiplication2}
		\label{algo:addition-and-multiplication-2}
		\begin{algorithmic}[1] 
			\Function{AdditionAndMultiplication2}{$n$}
			\State $s=0$
			\For{$i = 1$ to $n$}
				\For{$j = 1$ to $i$}
					\State $s = s + j \cdot (i - j + 1)$
				\EndFor
			\EndFor
			\EndFunction
		\end{algorithmic}
	\end{algorithm}	
	
	\bigskip	
	
	How many additions, multiplications, and copying?
\end{frame}

\begin{frame}{Example 2: How Many Additions and Multiplications (2/2)}

%	\begin{center}
%		\includegraphics[scale=0.275]{example2-nested-loop}
%	\end{center}
	
	Number of operations = $1 + (n+1) + \sum_{i=1}^{n}{\sum_{j=1}^{i}{5}} = 1+(n+1)+\frac{5}{2}(n^2+n)$
\end{frame}

\begin{frame}{Example 3: How Many Subtractions (1/2)}
		Assume $n$ is a positive integer.		
		\begin{algorithm}[H]
		\caption{Subtractions}
		\label{algo:subtractions}
		\begin{algorithmic}[1] 
			\Function{Subtractions}{$n$}
			\For{$i = \lfloor n/2 \rfloor $ to $n$}
				\State $a = n - i$
			\EndFor
			\EndFunction
		\end{algorithmic}
	\end{algorithm}	
	Compute the number of subtractions and copying that must be performed when the function is executed.
\end{frame}

\begin{frame}{Example 3: How Many Subtractions (2/2)}

	If $n$ is even, then $\lfloor \frac{n}{2} \rfloor = \frac{n}{2}$. \\
	Therefore, number of operations = $\frac{3n}{2}+4$.
	
	\bigskip	
	
	If $n$ is odd, then $\lfloor \frac{n}{2} \rfloor = \frac{n-1}{2}$. \\
	Therefore, number of operations = $\frac{3n}{2}+\frac{9}{2}$.
\end{frame}

\section{Guiding Principles for the Analysis of Algorithms}
\begin{frame}{Prinsip untuk Menganalisis Algoritma}
	\begin{enumerate}
		\item Prinsip \#1: Worst-Case Analysis 
		
		\bigskip		
		
		\item Prinsip \#2: Melihat Big Picture-nya
		
		\bigskip		
		
		\item Prinsip \#3: Analisis Asimtotik (\textit{Asymptotic Analysis})
	\end{enumerate}
\end{frame}

\begin{frame}{Prinsip \#1: Worst-Case Analysis}
	Selalu menghitung jumlah basic operations untuk kasus terjelek (\textit{worst-case scenario})
\end{frame}

\begin{frame}{The Sequential Search Algorithm (1/2)}
	\begin{figure}[!ht]
		\centering
		\includegraphics[scale=.26]{sequential-search-algorithm}
		\caption{Sequential search dari $a[1]$, $a[2]$, $\ldots$, $a[7]$ untuk mencari $x$ dengan $x = a[5]$}
		\label{fig:sequential-search-epps}
	\end{figure}
	Carilah jumlah operasi untuk the \textbf{best} and the \textbf{worst} scenario dari sequential search algorithm~\citep{epp2010discrete}.
\end{frame}

\begin{frame}{The Sequential Search Algorithm (2/2)}
	\textbf{Jawab}:\\
	Untuk \textbf{best} case is $1$ dan untuk \textit{worst-case} adalah $n$.		
\end{frame}

%\begin{frame}[fragile]{Calculating Time Complexity (1/2)}
%	\begin{minted}{python}
%def getMin(arr):
%    """
%    Get the smallest number on an array of numbers.
%    
%    Parameter
%    ---------
%    arr : array of numbers
%    """
%    minimum = arr[0] # for the sake of simplicity
%    for indeks, elemen in enumerate(arr):
%        if arr[indeks] < minimum: 
%            minimum = arr[indeks]  
%    return minimum		
%	\end{minted}
%\end{frame}
%
%\begin{frame}{Calculating Time Complexity (2/2)}
%	\begin{itemize}
%		\item We can represent \texttt{getMin} as a function of the size of the input \texttt{n} based on the number of operations it has to perform.
%		\item For simplicity, let's assume that each line of code takes the same amount of time in the CPU to execute.
%		\item \emph{How many total operations are being executed by the function?}
%	\end{itemize}
%	Answer: \texttt{2(n) + 2} $\Longrightarrow$ too specific and hard to compare algorithms with it.
%	
%	\bigskip
%	
%	We apply the \textbf{asymptotic analysis} to simplify this expression further.		
%\end{frame}
%



\begin{frame}{Example 4: Insertion Sort (1/4)}
		\begin{algorithm}[H]
		\caption{Insertion Sort}
		\label{algo:insertion-sort}
		\begin{algorithmic}[1] 
			\Function{InsertionSort}{$a$}
%			\Comment \textbf{Input}: \textit{An array $a$ with $n$ elements} \newline
%			\Comment \textbf{Output}
			\For{$k = 2$ to $n$}
				\State $x = a[k]$
				\State $j = k-1$
				\While{$j \neq 0$}
					\If{$x < a[j]$}
						\State $a[j+1] = a[j]$ 
						\State $a[j] = x$
						\State $j = j - 1$
					\Else
						\State $j=0$				
					\EndIf
				\EndWhile 
			\EndFor
%			\State $S = 0$
%			\For{$i = 1$ to $n$}
%			\State $S = S + i * i$
%			\EndFor 
%			\State \Return $S$ 
			\EndFunction
		\end{algorithmic}			
	\end{algorithm}	
\end{frame}

\begin{frame}{Example 4: Insertion Sort (2/4)}
	Given an array
	\begin{equation*}
		a[1]=6, \; a[2]=3, \; a[3]=5, \; a[4]=7, \; a[5]=2.
	\end{equation*}
	The table below shows the result.
	\begin{center}
		\includegraphics[scale=.25]{result-insertion-sort}
	\end{center}
\end{frame}

\begin{frame}{Example 4: Insertion Sort (3/4)}
	Construct a \textbf{trace table} showing the action of insertion sort on the array
	\begin{equation*}
		a[1]=6, \; a[2]=3, \; a[3]=5, \; a[4]=7, \; a[5]=2.
	\end{equation*}	
\end{frame}

\begin{frame}{Example 4: Insertion Sort (4/4)}
	\begin{center}
		\includegraphics[scale=.085]{trace-table-for-insertion-sort}
	\end{center}
\end{frame}

\begin{frame}{Finding a Worst-Case Order for Insertion Sort (1/2)}
	What is the number of comparisons that are performed when insertion sort is applied to the array $a[1], a[2], a[3], \ldots a[n]$?
\end{frame}

\begin{frame}{Finding a Worst-Case Order for Insertion Sort (2/2)}
		The maximum number of iteration of the \textbf{while} loop is $k$. Moreover, for each iteration, there is a double comparison. Therefore,
		\begin{align*}
			2 \cdot 2 + 2 \cdot 3 + \cdots + 2 \cdot n &= 2(2 + 3 + \cdots + n) \\
			                                           &= 2[(1+2+3 + \cdots + n) - 1] \\
			                                           &= 2\left( \frac{n(n+1)}{2} - 1 \right) \\
			                                           &= n(n+1) -2.			                                           
		\end{align*}	
\end{frame}

\begin{frame}{Prinsip \#2: Big-Picture Analysis (1/2)}
	Prinsip ini menyatakan bahwa faktor konstan atau suku yang berderajat rendah dapat dibuang pada saat \textit{running time}.
	
	\bigskip
	
	Contoh:
	\begin{align*}
		3x^3 &\approx x^3  \\
		2x^2+x &\approx x^2				
	\end{align*}		
\end{frame}

\begin{frame}{Prinsip \#2: Big-Picture Analysis (2/2)}
	\begin{align*}
		4x^{\frac{4}{3}}-15x + 7 &\approx ? \\
		\sqrt{x}(38x^{5} + 9) &\approx ? \\
		23x^4 + 8 x^2 + 4x &\approx ?
	\end{align*}
\end{frame}

\begin{frame}{Prinsip \#3: Asymptotic Analysis (1/3)}
	Fokus kepada \textit{rate of growth} dari algorithm's running time ketika input size $n$ bertambah besar.
	
	\bigskip	
	
	Contoh: \\
	Algoritma running time dari \textbf{merge sort} adalah $6n \log_{2}{n} + 6n$ operasi, sedangkan running time dari \textbf{insertion sort} adalah $\frac{1}{2}n^2$.

	\bigskip
	
	Merge sort \textit{lebih baik} daripada insertion sort. Apakah benar? 		
\end{frame}

\begin{frame}{Prinsip \#3: Asymptotic Analysis (2/3)}
	\begin{figure}[!ht]
		\centering
		\includegraphics[scale=.35]{merge-sort-small-values}
		\caption{Perbandingan ketika $n$ kecil}
	\end{figure}
\end{frame}

\begin{frame}{Prinsip \#3: Asymptotic Analysis (3/3)}
	\begin{figure}[!ht]
		\centering
		\includegraphics[scale=.35]{merge-sort-medium-values}
		\caption{Perbandingan ketika $n$ sedang}
	\end{figure}
\end{frame}

\begin{frame}{What is a "Fast" Algorithm?}
	
	\begin{center}
	\noindent\textbf{A "fast algorithm" is an algorithm whose worst-case running time grows slowly with the input size}\\\citep{roughgarden2017illuminatedpart1}
	\end{center}			
\end{frame}

\section{Asymptotic Notation}
\begin{frame}{Notasi Asimtotik (1/2)}
\begin{itemize}
	\item Asymptotic notation provides the basic vocabulary for discussing the design and analysis of algorithms~\citep{roughgarden2017illuminatedpart1}. 
	\item It's important that you know what programmers mean when they say that one piece of code runs in "big-O of $n$ time", while another runs in "big-O of $n$-squared time"~\citep{roughgarden2017illuminatedpart1}. 
\end{itemize}

\bigskip

\begin{center}
	\textit{suppress constant factors and lower-order terms}
\end{center}	
\end{frame}

\begin{frame}{Four Examples}
	Four example dapat dibaca di buku \citet{roughgarden2017illuminatedpart1} hlm. 38. 
\end{frame}


\begin{frame}{Notasi Asimtotik (2/2)}

	Running time merge sort 
	
	\begin{equation*}
		6n \log_2{n} + 6n
	\end{equation*}
	dapat dikatakan bahwa running time dari merge sort adalah "big-O of $n \log n$", ditulis $O(n \log n)$.

\end{frame}

\begin{frame}{Notasi Big-O}
	Notasi Big-O berhubungan dengan fungsi $T(n)$ yang didefinisikan pada bilangan bulat positif $n = 1, 2, \ldots$.
	
	\bigskip	
	
	$T(n)$ akan hampir selalu melambangkan batas atas \textit{running time} dari sebuah algoritma, sebagai fungsi dari $n$, ukuran input. 
	
	\bigskip
	
	\textbf{Notasi Big-O (versi bahasa Inggris)}
	\begin{center}
			$T(n) = O(f(n))$ if and only if $T(n)$ is eventually bounded above by a constant multiple of $f(n)$. 
	\end{center}		
	
	\textbf{Notasi Big-O (versi matematika)}
	\begin{center}
			$T(n) = O(f(n))$ if and only if there exist positive constants $c$ and $n_0$ such that
				\begin{equation*}
					T(n) \leq c \cdot f(n)
				\end{equation*}
			for all $n \geq n_0$.
	\end{center}			
\end{frame}

\begin{frame}{Ilustrasi Notasi Big-O}
	\begin{figure}[!ht]
		\centering
		\includegraphics[scale=.275]{ilustrasi-big-oh}
		\caption{Ilustrasi $T(n) = O(f(n))$~\citep{roughgarden2017illuminatedpart1}}
	\end{figure}
\end{frame}


\begin{frame}{Another Example}
	Let's say we have the following function: $T(n) = 3n^2+2n+20$. What would be the result of $T(n)$ using the asymptotic analysis?
	
	\bigskip
	
	\begin{block}{\textbf{The Answer}}
		$T(n) = O(n^2)$.
	\end{block}
	
	\begin{center}
		Mengapa?
	\end{center}
\end{frame}



\begin{frame}{Most Common Running Times (1/2)}
	 \textbf{Assuming}: 1 GHz CPU and that it can execute on average one instruction in 1 nanosecond (usually takes more time). Also, bear in mind that each line might be translated into dozens of CPU instructions depending on the programming language. \\
	 \textit{Let's see the most common running times and their relationship with time}:
	 \begin{center}
	 	\begin{tabular}{|l|l|l|l|l|}
	 		\hline
	 		$\bm{n}$ & $\bm{O(1)}$ & $\bm{O(\log n)}$ & $\bm{O(n)}$ & $\bm{O(n \log n)}$ \\
	 		\hline
	 		1 			& < 1 sec & < 1 sec 	& < 1 sec & < 1 sec \\
	 		\hline
	 		10 			& < 1 sec & < 1 sec 	& < 1 sec & < 1 sec \\
	 		\hline
	 		100 			& < 1 sec & < 1 sec 	& < 1 sec & < 1 sec \\
	 		\hline
	 		1,000 		& < 1 sec & < 1 sec 	& < 1 sec & < 1 sec \\
	 		\hline
	 		10,000 		& < 1 sec & < 1 sec 	& < 1 sec & < 1 sec \\
	 		\hline
	 		100,000 		& < 1 sec & < 1 sec 	& < 1 sec &  1 sec \\
	 		\hline
	 		1,000,000 	& < 1 sec & < 1 sec 	& 1 sec &  20 sec \\
	 		\hline
	 	\end{tabular}
	 \end{center}
\end{frame}

\begin{frame}{Most Common Running Times (2/2)}
	 \begin{center}
	 	\begin{tabular}{|l|l|l|l|}
	 		\hline
	 		$\bm{n}$ & $\bm{O(n^2)}$ & $\bm{O(2^n)}$ & $\bm{O(n!)}$  \\
	 		\hline
	 		1 			& < 1 sec 	& < 1 sec 				& < 1 sec  \\
	 		\hline
	 		10 			& < 1 sec 	& < 1 sec 				& 4 sec  \\
	 		\hline
	 		100 			& < 1 sec 	& 40,170 trillion years 	& > vigintillion years  \\
	 		\hline
	 		1,000 		& < 1 sec 	& > vigintillion years 	& > centillion years  \\
	 		\hline
	 		10,000 		& 2 min 		& > centillion years 	& > centillion years  \\
	 		\hline
	 		100,000 		& 3 hours 	& > centillion years 	& > centillion years  \\
	 		\hline
	 		1,000,000 	& 12 days 	& > centillion years 	& > centillion years  \\
	 		\hline
	 	\end{tabular}
	 \end{center}
	Trillion = $10^{12}$ \\ 
	Vigintillion = $10^{63}$ \\
	Centillion = $10^{303}$
\end{frame}

\section{8 Time Complexities that Every Programmer Should Know}
\begin{frame}{8 Time Complexities that Every Programmer Should Know}
	Please refer to the \emph{Jupyter Notebook}. 

	\bigskip
	
	The notebook is hosted at \href{https://bitbucket.org/hbunyamin/demo-complexity-for-discrete-mathematics-class/src/master/}{\emph{\color{blue} a bitbucket repo}}.
\end{frame}










%\begin{frame}{Blocks}
%\begin{block}{Block Title}
%You can also highlight sections of your presentation in a block, with it's own title
%\end{block}
%\begin{theorem}
%There are separate environments for theorems, examples, definitions and proofs.
%\end{theorem}
%\begin{example}
%Here is an example of an example block.
%\end{example}
%\end{frame}


% All of the following is optional and typically not needed. 
\appendix
\section<presentation>*{\appendixname}
\subsection<presentation>*{For Further Reading}

\begin{frame}[allowframebreaks]
  \frametitle<presentation>{Daftar Pustaka}
    {\footnotesize
    \bibliographystyle{apalike}
    \bibliography{references}
    }    
\end{frame}

%\makeatletter % to change template
%    \setbeamertemplate{headline}[default] % not mandatory, but I though it was better to set it blank
%    \def\beamer@entrycode{\vspace*{-\headheight}} % here is the part we are interested in :)
%\makeatother

\begin{frame}[plain]
		\centering\includegraphics[scale=0.5]{Logo-Maranatha-Untuk-Belakang-02}	
\end{frame}

\end{document}



\documentclass[pdf]{beamer}

%\usepackage{lmodern}
\usepackage[font=scriptsize,skip=0pt,justification=justified,singlelinecheck=false]{caption}
%\usepackage{enumitem}
\usepackage{natbib}
\usepackage{bm}
\usepackage{mathtools}
\usepackage[makeroom]{cancel}

\usepackage{xcolor}
\usepackage{hyperref}

% Using underline
\usepackage{soul}

\usepackage{algorithm}
\usepackage{algpseudocode}
%\usepackage[ruled,vlined]{algorithm2e}
%\usepackage{clrscode3e}


%\usepackage{adjustbox}

%remove the icon
\setbeamertemplate{bibliography item}{}

%remove line breaks
\setbeamertemplate{bibliography entry title}{}
\setbeamertemplate{bibliography entry location}{}
\setbeamertemplate{bibliography entry note}{}
% Use number for caption
\setbeamertemplate{caption}[numbered]{}

\newtheorem{mydef}[theorem]{\Large \underline{\textbf{Definisi}}}

\makeatletter
\def\th@mystyle{%
    \normalfont % body font
    \setbeamercolor{block title example}{bg=blue,fg=white}
    \setbeamercolor{block body example}{bg=blue!20,fg=black}
    \def\inserttheoremblockenv{exampleblock}
  }
\makeatother
\theoremstyle{mystyle}
\newtheorem*{remark}{\textbf{Definition}}


% This file is a solution template for:

% - Talk at a conference/colloquium.
% - Talk length is about 20min.
% - Style is ornate.

% Copyright 2004 by Till Tantau <tantau@users.sourceforge.net>.
%
% In principle, this file can be redistributed and/or modified under
% the terms of the GNU Public License, version 2.
%
% However, this file is supposed to be a template to be modified
% for your own needs. For this reason, if you use this file as a
% template and not specifically distribute it as part of a another
% package/program, I grant the extra permission to freely copy and
% modify this file as you see fit and even to delete this copyright
% notice.  


\mode<presentation>
{
%  \usetheme{AnnArbor} % 
%	\usetheme{Frankfurt}
   \usetheme{Madrid}
  % or ...

%  \setbeamercovered{transparent}
  % or whatever (possibly just delete it)
}


\usepackage[english]{babel}
% or whatever

\usepackage[latin1]{inputenc}
% or whatever

\usepackage{times}
\usepackage[T1]{fontenc}
\usepackage{wasysym}

\usepackage{graphicx} % Necessary to use \scalebox

% Define absolute and norm
\DeclarePairedDelimiter\abs{\lvert}{\rvert}%
\DeclarePairedDelimiter\norm{\lVert}{\rVert}%

% ==============================
%       Redefine emphasize
% ==============================
\let\emph\relax % there's no \RedeclareTextFontCommand
\DeclareTextFontCommand{\emph}{\bfseries\em}


% Swap the definition of \abs* and \norm*, so that \abs
% and \norm resizes the size of the brackets, and the 
% starred version does not.
\makeatletter
\let\oldabs\abs
\def\abs{\@ifstar{\oldabs}{\oldabs*}}
%
\let\oldnorm\norm
\def\norm{\@ifstar{\oldnorm}{\oldnorm*}}
\makeatother

\usepackage{color}
\definecolor{myblue}{rgb}{.8,.8,1}
\usepackage{empheq}
% Or whatever. Note that the encoding and the font should match. If T1
% does not look nice, try deleting the line with the fontenc.

\newlength\mytemplen
\newsavebox\mytempbox

\makeatletter
\newcommand\mybluebox{%
    \@ifnextchar[%]
       {\@mybluebox}%
       {\@mybluebox[0pt]}}

\def\@mybluebox[#1]{%
    \@ifnextchar[%]
       {\@@mybluebox[#1]}%
       {\@@mybluebox[#1][0pt]}}

\def\@@mybluebox[#1][#2]#3{
    \sbox\mytempbox{#3}%
    \mytemplen\ht\mytempbox
    \advance\mytemplen #1\relax
    \ht\mytempbox\mytemplen
    \mytemplen\dp\mytempbox
    \advance\mytemplen #2\relax
    \dp\mytempbox\mytemplen
    \colorbox{myblue}{\hspace{1em}\usebox{\mytempbox}\hspace{1em}}}

\makeatother

\title[AI (Now \& Future )] % (optional, use only with long paper titles)
{\textbf{AI Now \& Future}}

\subtitle
{\textit{Jalur Peminatan: AI Specialist}}

\author[Hendra Bunyamin] % (optional, use only with lots of authors)
{Hendra Bunyamin}
%{F.~Author\inst{1} \and S.~Another\inst{2}} --> original
% - Give the names in the same order as the appear in the paper.
% - Use the \inst{?} command only if the authors have different
%   affiliation.

\institute[ ] % (optional, but mostly needed)
{
%  \inst{1}%
  Program Studi Teknik Informatika\\
  Fakultas Teknologi Informasi\\
  Universitas Kristen Maranatha
%  \and
%  \inst{2}%
%  Department of Theoretical Philosophy\\
%  University of Elsewhere
}
% - Use the \inst command only if there are several affiliations.
% - Keep it simple, no one is interested in your street address.

%\date[CFP 2003] % (optional, should be abbreviation of conference name)
%{Conference on Fabulous Presentations, 2003}
% - Either use conference name or its abbreviation.
% - Not really informative to the audience, more for people (including
%   yourself) who are reading the slides online

\subject{Sibernetika}
% This is only inserted into the PDF information catalog. Can be left
% out. 

% If you have a file called "university-logo-filename.xxx", where xxx
% is a graphic format that can be processed by latex or pdflatex,
% resp., then you can add a logo as follows:

\pgfdeclareimage[height=1.5cm]{university-logo}{logo-mcu}
\logo{\pgfuseimage{university-logo}}


% Delete this, if you do not want the table of contents to pop up at
% the beginning of each subsection:
\AtBeginSection[]
{
  \begin{frame}<beamer>{Outline}
    \tableofcontents[currentsection,currentsection]
  \end{frame}
}


% If you wish to uncover everything in a step-wise fashion, uncomment
% the following command: 

%\beamerdefaultoverlayspecification{<+->}

\begin{document}

\begin{frame}
  \titlepage
\end{frame}

\begin{frame}{Outline}
  \tableofcontents
  % You might wish to add the option [pausesections]
\end{frame}


% Structuring a talk is a difficult task and the following structure
% may not be suitable. Here are some rules that apply for this
% solution: 

% - Exactly two or three sections (other than the summary).
% - At *most* three subsections per section.
% - Talk about 30s to 2min per frame. So there should be between about
%   15 and 30 frames, all told.

% - A conference audience is likely to know very little of what you
%   are going to talk about. So *simplify*!
% - In a 20min talk, getting the main ideas across is hard
%   enough. Leave out details, even if it means being less precise than
%   you think necessary.
% - If you omit details that are vital to the proof/implementation,
%   just say so once. Everybody will be happy with that.

%\begin{frame}{Make Titles Informative. Use Uppercase Letters.}{Subtitles are optional.}

\section{Motivasi AI}
\begin{frame}{Introduction}
	\begin{figure}[!ht]
		\centering
		\includegraphics[scale=.225]{AI-value-creation}
		\caption{Source: McKinsey Global Institute~\citep{ng2019AIForEveryone}}
		\label{fig:ai-value-creation}
	\end{figure}
	\begin{align*}
	\$13 \text{ trillion } &= \$13 \times 10^{12} \\
	                       &= \text{Rp}183.000.000.000.000.000,- \\
                           &= 183 \text{ billiard}.	                       	
	\end{align*}	  
\end{frame}

\begin{frame}{Demystifying AI}
	Artificial Intelligence or \textbf{AI} can be divided into 2 as follows:
	\begin{itemize}
		\item<2-> \textbf{ANI} $\Rightarrow$ \textit{Artificial Narrow Intelligence}. \\
		\textbf{Examples}: smart speaker, self-driving car, web search, AI in farming and factories.
		\bigskip
		\item<3-> \textbf{AGI} $\Rightarrow$ \textit{Artificial General Intelligence}. \\
		\textbf{Examples}: Do anything or \textbf{even more} than a human can do.
	\end{itemize}
	\begin{center}
		\includegraphics<3->[scale=.125]{ex-machina} \qquad \includegraphics<3->[scale=.125]{upgrade}
	\end{center}
\end{frame}

\section{Machine Learning}
\begin{frame}{Machine Learning (1/2)}
	\begin{itemize}
		\item<2-> One of the tools that drive the significant progress of AI is \textbf{Machine Learning} (ML).
		\item<3-> \textbf{Machine Learning} is a set of methods that allow computers to \textit{learn from data} to \textit{make and improve predictions}, e.g., cancer, weekly sales, credit default~\citep{molnar2019}.  
		\begin{figure}[!ht]
			\centering
			\includegraphics<4->[scale=0.18]{images/programming-with-without-ml}
			\caption{\onslide<4-> A paradigm shift from "normal programming" to "indirect programming"}
		\end{figure}
	\end{itemize}		
\end{frame}

\begin{frame}{Machine Learning (2/2)}	
	The way a machine or computer learns can be categorized into several types~\citep{ng2023generative}:
	\begin{itemize}
		\item<2-> \textcolor <7-> {orange} {\textbf<7->{Supervised Learning}},
		\item<3-> Unsupervised Learning, and
		\item<4-> Reinforcement learning
		\item<5-> Generative AI
	\end{itemize}	
	\begin{center}
		\includegraphics<6->[scale=.32]{images/diagram-learning.png}
	\end{center}
	
	
\end{frame}

\begin{frame}{Supervised Learning (1/3)}
	\begin{itemize}		
		\item A common type of of Machine Learning is a type of AI that learns from $A$ to $B$	or is often called \emph{Supervised Learning}.
		
		\bigskip
		
		\begin{center}
			
			\scalebox{2}{
				$A \longrightarrow B$
			} \\	
			\scalebox{1}{
				\qquad input \qquad \; \; output
			}							
		\end{center}
	\end{itemize}	
\end{frame}

\begin{frame}{Supervised Learning (2/3)}
	\begin{table}[!ht]
		\centering
		\begin{tabular}{ccc|l}
			\textbf{Input ($\bm{A}$)} &  & \textbf{Output ($\bm{B}$)} & \textbf{Application} \\
			\hline
			\onslide<2-> Email     & $\longrightarrow$  & spam? (0/1)                & \textbf{Spam filtering}   \\
			&                    &                            &                  \\
			\hline         
			\onslide<3-> Audio recording    & $\longrightarrow$  & text transcript            & \textbf{Speech}      \\
			&                    &                            &      \textbf{recognition}            \\
			\hline      
			\onslide<4-> Ad, user info & $\longrightarrow$ & Click? (0/1)            & \textbf{Online advertising} \\
			&                    &                               &                  \\
			\hline    
			\onslide<5-> Image, radar info & $\longrightarrow$ & Position of other cars &  \textbf{Self-driving car} \\
			&                    &                               &                  \\
			\hline
			\onslide<6-> Image of phone & $\longrightarrow$ & Defect? (0/1)             & \textbf{Visual inspection} \\ 
			&                    &                               &                  \\
			\hline   
			\onslide<7-> Restaurant reviews & $\longrightarrow$ & Sentiment (pos/neg)             & \textbf{Reputation} \\ 		    		 			
			&                    &                               &           \textbf{Monitoring}        \\
			\hline 		                                                      
			\onslide<8-> X-ray image & $\longrightarrow$ & Diagnosis             & \textbf{Healthcare} \\ 
			\hline           			             
		\end{tabular}
	\end{table}
\end{frame}


\begin{frame}{Supervised Learning (3/3)}
	Consider the following examples \\
	(\textit{input} $A$ in \textbf{bold} and \textit{output} $B$ in italic)~\citep{trask2019grokking}:
	\begin{itemize}
		\item<2-> Using \textbf{pixels} of an image to detect the \textit{presence or absence of a cat}
		\item<3-> Using the \textbf{movies you've liked} to predict more \textit{movies you may like}
		\item<4-> Using someone's \textbf{words} to predict whether they're \textit{happy} or \textit{sad} 
		\item<5-> Using \textbf{weather sensor data} to predict the \textit{probability of rain}
		\item<6-> Using \textbf{car engine sensors} to predict the optimal tuning \textit{settings}
		\item<7-> Using \textbf{news data} to predict tomorrow's stock \textit{price}
		\item<8-> Using a raw \textbf{audio file} to predict a \textit{transcript} of the audio.
	\end{itemize}	
\end{frame}

\begin{frame}{Why Now?}
	\begin{figure}[!ht]
		\centering
		\includegraphics[scale=.2]{images/big-data}
		\caption{Large neural net + Big Data = High Performance \citep{ng2023generative}}
		\label{fig:big-data}
	\end{figure}
\end{frame}

\begin{frame}{Example of a Table of Data (Dataset) (1/3)}
	\begin{table}[!ht]
		\centering
		\includegraphics[scale=.3]{example-dataset-2}
		\caption{House prices dataset~\citep{ng2019AIForEveryone}}
		\label{fig:example-dataset-2}
	\end{table}
\end{frame}

\begin{frame}{Example of a Table of Data (Dataset) (2/3)}
	\begin{table}[!ht]
		\centering
		\includegraphics[scale=.3]{example-dataset-1}
		\caption{House prices dataset~\citep{ng2019AIForEveryone}}
		\label{fig:example-dataset-1}
	\end{table}
\end{frame}

\begin{frame}{Example of a Table of Data (Dataset) (3/3)}
	\begin{table}[!ht]
		\centering
		\includegraphics[scale=.3]{example-dataset-3}
		\caption{Cat images dataset~\citep{ng2019AIForEveryone}}
		\label{fig:example-dataset-3}
	\end{table}
\end{frame}

\begin{frame}{Acquiring data}
	\begin{itemize}
		\item<2-> Manual labeling
		\begin{center}
			\includegraphics[scale=.3]{manual-labeling}
		\end{center}
		\item<3-> From observing behaviors
		\begin{center}
			\includegraphics[scale=.35]{observing-behaviors}
		\end{center}
		\item<4-> Download from websites / partnerships \\
		Contoh: \href{https://www.kaggle.com/datasets}{\textcolor{orange}{\underline{ Kaggle}}}, \href{http://archive.ics.uci.edu/ml/datasets.php}{\textcolor{blue}{\underline{UCI Machine Learning Repo}}}		
	\end{itemize}
\end{frame}

\begin{frame}{Data is Messy}
	\begin{itemize}
		\item<2-> Garbage in, garbage out
		\item<3-> Data problems: \textit{incorrect labels} and \textit{missing values} 
		\begin{center}
			\includegraphics[scale=.3]{data-problems} 
		\end{center}
		\item<4-> Multiple types of data \\
		\textit{images}, \textit{audio}, \textit{text} $\Rightarrow$ \textbf{unstructured data}					
	\end{itemize}
\end{frame}

\section{Machine Learning vs. Data Science}
\begin{frame}{AI, Machine Learning, and Data Science}
	\begin{figure}[!ht]
		\centering
		\includegraphics[scale=.45]{diagram-venn-deep-learning}
		\caption{Relationship among AI, ML, DL, and DS~\citep{kharkovyna2019ABeginnersGuide}}
	\end{figure}
\end{frame}

\begin{frame}{Machine Learning vs. Data Science (1/2)}
	\begin{figure}[!ht]
		\centering
		\includegraphics[scale=.25]{ml-vs-ds}		
		\caption{Home prices~\citep{ng2019AIForEveryone}}
	\end{figure}
	\begin{itemize}
		\item<2-> \textbf{According to \textit{Machine Learning}}: \\
	$A \longrightarrow B$: Running AI system (e.g., websites / mobile app)
		\item<3-> 	\textbf{According to \textit{Data Science}}: \\
	\textit{Homes with 3 bedrooms are more expensive than homes with 2 bedrooms of a similar size}. \\		
	\textit{Newly renovated homes have a 15\% premium}.		
	\end{itemize}
%	\vspace*{-.5cm}			
\end{frame}

\begin{frame}{Machine Learning vs. Data Science (2/2)}
	\begin{table}[!ht]
		\centering
		\begin{tabular}{cc}
			\textbf{Machine Learning}        & \textbf{Data Science} \\
			                                 &                       \\
			 \onslide<2-> "\textit{Field of study that gives}      & \onslide<3-> \textit{Science of extracting knowledge} \\
			 \onslide<2-> \textit{computers the ability to learn}   & \onslide<3-> \textit{and insights from data.} \\
			 \onslide<2-> \textit{without being explicitly}         &   \\
			 \onslide<2-> \textit{programmed.}"                     &   \\
			 \onslide<2-> $\longrightarrow$ \textbf{software}       & \onslide<3-> $\longrightarrow$ \textbf{slide presentation} or \textbf{report}  \\
			 \onslide<2-> -Arthur Samuel (1959)           & 
		\end{tabular}
	\end{table}
\end{frame}

\section{Deep Learning}
\begin{frame}{Dataset: Home Prices}
		\begin{figure}[!ht]
		\centering
		\includegraphics[scale=.25]{ml-vs-ds}		
		\caption{Home prices~\citep{ng2019AIForEveryone}}
	\end{figure}
\end{frame}

\begin{frame}{Deep Learning (1/3)}
	\begin{center}
		\includegraphics[scale=.275]{deep-learning}
	\end{center}	
\end{frame}

\begin{frame}{Deep Learning (2/3)}
		\begin{figure}[!ht]
	\centering
	\includegraphics[scale=.4]{images/brain-neurons}		
	\caption{Neuron-neuron di Otak \citep{ankrom2020howbrain}}
\end{figure}
\end{frame}

\begin{frame}{Deep Learning (2/3)}
	\begin{center}
		\includegraphics[scale=.24]{deep-learning-2}
	\end{center}	
\end{frame}

\begin{frame}{Demand prediction (1/2)}
	\begin{columns}[c]		
		\begin{column}{0.6\textwidth}
			\hspace{20pt}
			\includegraphics[scale=.225]{demand-prediction}
		\end{column}
		\hspace{-10pt}
		\begin{column}{0.4\textwidth}
			\includegraphics[scale=.225]{t-shirt}				
		\end{column}		
	\end{columns}	
\end{frame}

\begin{frame}{Demand prediction (2/2)}
	\centering
	\includegraphics[scale=.225]{demand-prediction-line}
\end{frame}

\begin{frame}{Demand prediction: a little bit more complex (1/4)}
	\centering
	\includegraphics[scale=.275]{demand-prediction-nn-0.png}
\end{frame}

\begin{frame}{Demand prediction: a little bit more complex (2/4)}
	\centering
	\includegraphics[scale=.275]{demand-prediction-nn.png}
\end{frame}

\begin{frame}{Demand prediction: a little bit more complex (3/4)}
	\centering
	\includegraphics[scale=.275]{demand-prediction-nn-2.png}
\end{frame}

\begin{frame}{Demand prediction: a little bit more complex (4/4)}
	\centering
	\includegraphics[scale=.275]{demand-prediction-nn-3.png}
\end{frame}


\begin{frame}{NN Application: Face recognition (1/3)}
	We want to build a system that recognizes people from pictures.
	\begin{figure}[!ht]
		\centering
		\includegraphics[scale=.25]{face-recognition-intro}
		\caption{What a computer sees from an image (assume the picture is grayscale)~\citep{ng2019AIForEveryone}}
	\end{figure}
	
\end{frame}

\begin{frame}{NN Application: Face recognition (2/3)}
		\centering
		\includegraphics[scale=.25]{face-recognition-intro-2}
\end{frame}

\begin{frame}{NN Application: Face recognition (3/3)}
		\centering
		\includegraphics[scale=.225]{face-recognition-intro-3}
\end{frame}

\begin{frame}{How Does a Neural Network Learn?}
	Watch \href{https://phiresky.github.io/neural-network-demo/}{\textcolor{pink}{\textbf{a Demo}}} by~\citet{phiresky2017neuralnetwork}.
\end{frame}


\section{What Machine Learning Can and Cannot Do}
\begin{frame}{Supervised Learning}
	\begin{center}
		\includegraphics[scale=.3]{what-ML-can-do}
	\end{center}
	\begin{center}
		\pause \textit{Anything you can do with 1 second of thought, we can probably now or soon automate.}
	\end{center}
\end{frame}

\begin{frame}{What machine learning today can and cannot do}
	You ordered a toy. The toy arrived late. Therefore, you write an email:
	
	\bigskip
	\textit{The toy arrived two days late, so I wasn't able to give it to my niece for her birthday.} \\
	\textit{Can I return it?}

	\bigskip
	\pause \textbf{Machine Learning Can Do}: \\
	$\longrightarrow$ "\textit{Refund request}" \\
	Input text $\longrightarrow$ Refund/Shipping/Other \\
	$A \longrightarrow B$
	
	\bigskip
	\pause \textbf{Machine Learning Cannot Do Elegantly Yet}: \\
	$\longrightarrow$ "\textit{Oh, sorry to hear that. I hope your niece had a good birthday. Yes, we can help with ...}"	
	
\end{frame}

\begin{frame}{What happens if you try?}
	\begin{tabular}{lcl}
		\textbf{Input (A)} & $\longrightarrow$ & \textbf{Output (B)} \\
		User email	       &                   & 2-3 paragraph response
	\end{tabular}

	\bigskip
	
	\textbf{Train Data}: 1000 examples 
	
	\bigskip
	
	\begin{tabular}{lcl}
		\onslide<2-> "My box was damaged" & $\longrightarrow$ & Thank you for your email. \\
		   	                 &                     &   \\
		\onslide<3-> "Where do I write a review?" & $\longrightarrow$ & Thank you for your email. \\		   	                 
		   	                 &                     &   \\
		\onslide<4-> "What's the return policy" & $\longrightarrow$ & Thank you for your email. \\		   	                 		   	                 		   	                 &                     &   \\
		\onslide<5-> "When is my box arriving?" & $\longrightarrow$ & Thank yes now your....
	\end{tabular}
\end{frame}
	
\begin{frame}{What makes an ML problem easier}
	\begin{enumerate}
		\item<2-> Learning a "simple" concept \\
		\begin{center}
			
			\scalebox{2}{
				$\leq 1 \text{ sec}$
			} 							
		\end{center}		
		
		\bigskip		
		
		\item<3-> Lots of data available \\
		\begin{center}
			
			\scalebox{2}{
				$A \longrightarrow B$
			} \\	
			\scalebox{1}{
				\qquad input \qquad \; \; output
			}							
		\end{center}				
	\end{enumerate}
\end{frame}	

\begin{frame}{Self-driving car}
	\begin{columns}[c]		
		\begin{column}{0.5\textwidth}
			\includegraphics<2->[scale=.225]{kiri-can}
			\vspace*{1.75cm}
		\end{column}
		\hspace{-50pt}
		\vrule{}
		\begin{column}{0.5\textwidth}
			\includegraphics<3->[scale=.225]{kanan-can}				
	\begin{enumerate}
		\item<4-> Data 
		\item<5-> Need high accuracy
	\end{enumerate}					
		\end{column}		
	\end{columns}	
\end{frame}
	
\begin{frame}{X-ray diagnosis}
	\begin{center}
		\includegraphics[scale=.225]{x-ray}	
	\end{center}		
	\centering
	\begin{tabular}{l|l}
		\multicolumn{1}{c|}{\textbf{Can do}} & \multicolumn{1}{c}{\textbf{Cannot do}} \\
		                                                 &   \\
	    \onslide<2->Diagnose pneumonia from                      &  \onslide<3->  Diagnose pneumonia from \\
	    \onslide<2->$\thicksim$10,000 labeled images             &  \onslide<3->  10 images of medical textbook \\
	                                                 &    \onslide<3->chapter explaining pneumonia	                  
	\end{tabular}	
\end{frame}

\begin{frame}{Strengths and weaknesses of machine learning}
	ML tends to work well when:
	\begin{enumerate}
		\item<2-> Learning a "simple" concept 
		\item<3-> There are lots of data available
	\end{enumerate}
	
	\bigskip	
	
	ML tends to work poorly when:
	\begin{enumerate}
		\item<4-> Learning complex concepts from small amounts of data
		\item<5-> It is asked to perform on new types of data
	\end{enumerate}
	\begin{center}
		\includegraphics<5->[scale=.275]{new-types-of-data}
	\end{center}
\end{frame}

\begin{frame}{Machine Learning}	
	The way a machine or computer learns can be categorized into several types~\citep{geron2019handson}:
	\begin{itemize}
		\item<2-> Supervised Learning,
		\item<3-> \textcolor <4-> {orange} {\textbf<4->{Unsupervised Learning}}.
	\end{itemize}	
	
\end{frame}

\begin{frame}{Unsupervised learning (1/2)}
	Clustering potato chip sales
	\begin{center}
		\includegraphics[scale=.25]{potato-chip-sales} \qquad \includegraphics<2->[scale=.25]{potato-chip-sales-2}
	\end{center}
	
\end{frame}

\begin{frame}{Unsupervised learning (2/2)}
	\textbf{Unsupervised learning}:
	\begin{center}
		\textit{Given data (without any specific desired output labels), find something interesting about the data.}
	\end{center}
	
	\textbf{Another example of unsupervised learning:}
	
	\bigskip	
	
	\pause Finding cats from unlabeled YouTube videos
	\begin{center}
		\includegraphics[scale=.25]{google-cats}
	\end{center}	
\end{frame}

\section{Jalur Peminatan AI: Becoming AI Specialist}
\begin{frame}{Mata Kuliah-Mata Kuliah Jalur Peminatan \textit{AI Specialist}}
	\begin{enumerate}
		\item \textit{Computer Vision}
		\bigskip
		\item \textit{Natural Language Processing}
		\bigskip
		\item \textit{AI Computing Platform}
	\end{enumerate}	
\end{frame}

\begin{frame}{Computer Vision (1/7)}
	\begin{itemize}
		\item Image classification/Object recognition 
		\begin{center}
			\includegraphics[scale=.4]{image-classification}	
		\end{center}
	\end{itemize}
\end{frame}

\begin{frame}{Computer Vision (2/7)}
	\begin{itemize}
		\item Face verification \& face identification
	\end{itemize}	
	\begin{figure}[!ht]
	\centering
	\includegraphics[scale=.232]{images/face-identification-recognition.jpg}
	\caption{Example of face verification (left) and face recognition (right) \citep{elgendy2020deeplearning4vision}}
\end{figure}
\end{frame}

\begin{frame}{Computer Vision (3/7)}
	\begin{itemize}
		\item Object detection
		\begin{center}
			\includegraphics[scale=.4]{object-detection-1} \qquad \includegraphics<2->[scale=.4]{object-detection-2}
			\bigskip			
			\includegraphics<3->[scale=.4]{object-detection-3}	
		\end{center}
	\end{itemize}
\end{frame}

\begin{frame}{Computer Vision (4/7)}
	\begin{itemize}
		\item Image Segmentation
		\begin{center}
			\includegraphics[scale=.4]{object-detection-1} \qquad \includegraphics<2->[scale=.4]{image-segmentation}	
		\end{center}
		\item<3-> Tracking
		\begin{center}
			\includegraphics[scale=.4]{tracking}
		\end{center}
	\end{itemize}
\end{frame}

\begin{frame}{Computer Vision (5/7)}
	\begin{itemize}
		\item Membuat gambar baru dari style pelukis ternama.
	\end{itemize}	
\begin{figure}[!ht]
	\centering
\includegraphics[scale=.2]{images/generating-style}
\caption{Style transfer from Van Goghs The Starry Night onto the original image \citep{elgendy2020deeplearning4vision}}
\end{figure}		
\end{frame}

\begin{frame}{Computer Vision (6/7)}
	\begin{itemize}
		\item Membuat gambar baru dari deskripsi teks
	\end{itemize}
	\begin{figure}[!ht]
	\centering
	\includegraphics[scale=.225]{images/gans+text+cv.jpg}
	\caption{StackGAN use a textual description of an object to generate a high-resolution image of the object matching that description \citep{elgendy2020deeplearning4vision}}
\end{figure}		
\end{frame}

\begin{frame}{Computer Vision (7/7)}
	\begin{itemize}
	\item Image recommendation system
\end{itemize}
\begin{figure}[!ht]
	\centering
	\includegraphics[scale=.225]{images/image-recommendation.jpg}
	\caption{Apparel search \citep{elgendy2020deeplearning4vision}}
\end{figure}		
\end{frame}

\begin{frame}{Natural Language Processing (1/8)}
	\begin{itemize}		
		\item Information search atau information retrieval \citep{kochmar2022getting}
	\end{itemize}
	\textbf{Contoh 1:}
	\begin{center}
		\includegraphics[scale=.25]{images/search-engine-1}
	\end{center}
	\textbf{Contoh 2:}
	\begin{center}
		\includegraphics[scale=.2]{images/search-engine-2}
	\end{center}
\end{frame}

\begin{frame}{Natural Language Processing (2/8)}
	\begin{itemize}		
		\item Advanced information search: Asking the machine precise questions
	\end{itemize}
	\begin{center}
		\includegraphics[scale=.225]{images/search-engine-3}
	\end{center}
\end{frame}

\begin{frame}{Natural Language Processing (3/8)}
The search for factual information on Google returns both the precise answer to the question and the accompanying explanation.
	\begin{center}
		\includegraphics[scale=.25]{images/search-engine-4}
	\end{center}
\end{frame}

\begin{frame}{Natural Language Processing (4/8)}
	\begin{itemize}
		\item Conversational agents dan intelligent virtual assistants
	\end{itemize}
	\begin{center}
		\includegraphics[scale=.225]{images/conversational-1}
	\end{center}
\end{frame}

\begin{frame}{Natural Language Processing (5/8)}
	\begin{itemize}
		\item Text prediction dan language generation
	\end{itemize}
	\begin{figure}[!ht]
		\centering
		\includegraphics[scale=.2]{images/text-generation}
		\caption{On the right: Google's Smart Reply for emails}
	\end{figure}
\end{frame}

\begin{frame}{Natural Language Processing (6/8)}
	\begin{itemize}
		\item Text Classification
		
		\begin{itemize}
			\item<2-> Sentiment recognition			
			\begin{tabular}{lcl}
				\onslide<3-> "The food was good"     & $\longrightarrow$ & \includegraphics[scale=.25]{four-stars}	\\
				\onslide<4-> "Service was horrible"  & $\longrightarrow$ & \includegraphics[scale=.25]{one-star}                      
			\end{tabular}
		
		\bigskip
			\item<5-> Spam/Hoax filtering			
			
		\bigskip	
			\item<6-> News article classification into: politics, business, sports, and so on.
		\end{itemize}		
	\end{itemize}
\end{frame}


\begin{frame}{Natural Language Processing (7/8)}
	\begin{itemize}
		\item \textbf{Machine translation} 	\\
	\textbf{Contoh 1}:\\
"AI adalah listrik baru" $\Longrightarrow$ "AI is new electricity"

\bigskip
\textbf{Contoh 2}:
\begin{figure}[!ht]
	\centering
	\includegraphics[scale=.2]{images/phrase-translation}
	\caption{Phrase translation between English and French}
\end{figure}
	\end{itemize}
\end{frame}

\begin{frame}{Natural Language Processing (8/8)}
	\begin{itemize}
		\item \textbf{Spell dan grammar checking}
	\end{itemize}
	\begin{figure}[!ht]
		\centering
		\includegraphics[scale=.2]{images/edit-distance}
		\caption{Possible corrections for the misspelling \textit{thougt}}
	\end{figure}
\end{frame}

\begin{frame}{NLP: Speech (1/2)}
	\begin{center}
		\includegraphics[scale=.25]{speech-1}
	\end{center}
	\begin{itemize}
		\item Speech recognition (speech-to-text) \\
		\includegraphics[scale=.2]{smart-speakers}
		\item<2-> Trigger word/wakeword detection \\
		Audio $\longrightarrow$ "Hey device"? (0/1)
	\end{itemize}
\end{frame}

\begin{frame}{NLP: Speech (2/2)}
	\begin{itemize}
		\item Speaker ID

		\bigskip

		\item<2-> Speech synthesis (text-to-speech, TTS)
		\begin{center}
			\texttt{The quick brown fox jumps over the lazy dog.}	
		\end{center}						
	\end{itemize}
\end{frame}

%\begin{frame}{Times Series Forecasting (1/3)}
%\begin{figure}[!ht]
%		\centering
%		\includegraphics[scale=.25]{images/time-series-forecasting}
%	\end{figure}		
%\end{frame}

%\begin{frame}{Times Series Forecasting (2/3)}
%	\onslide<2->{Forecasting has fascinated people for thousands of years, sometimes being considered a \textit{sign of divine inspiration}, and sometimes being seen as \textit{a criminal activity} \citep{hyndman2021forecasting}.}
%	\begin{figure}[!ht]
%		\centering
%		\includegraphics<3->[scale=.5]{fourexamples-time-series}
%	\end{figure}		
%\end{frame}
%
%\begin{frame}{Times Series Forecasting (3/3)}
%	This is an example of a time series which is decomposed by classical decomposition.
%	\begin{figure}[!ht]
%		\centering
%		\includegraphics<2->[scale=.4]{classical-empl-1}
%	\end{figure}		
%\end{frame}

%\begin{frame}{Times Series Forecasting (4/3)}
%	This is an example of a time series which is decomposed by a better decomposition, X-11.
%	\begin{figure}[!ht]
%		\centering
%		\includegraphics<2->[scale=.4]{x11-1}
%	\end{figure}		
%\end{frame}

%\begin{frame}{Times Series Forecasting (5/3)}
%	\begin{center}
%	\includegraphics[scale=.25]{images/prophet-symbol}
%	\end{center}
%	Salah satu algoritma untuk forecasting time series yang populer adalah \href{https://facebook.github.io/prophet/}{\textcolor{orange}{\textbf{Facebook Prophet}}} .	
%	\begin{center}
%		\includegraphics[scale=.35]{images/facebook-prophet-model}
%	\end{center}	
%\end{frame}

%\begin{frame}{Reinforcement learning (1/3)}
%	\begin{center}
%		\includegraphics[scale=.325]{reinforcement-learning}
%	\end{center}
%	Use a "reward" signal to tell the AI when it is doing well or poorly. It automatically learns to maximize its rewards. \\
%	(Contoh: \href{https://drive.google.com/file/d/19Zb2hjhyJ6lEUXo_OHQ0RSLrKSf6188x/view?usp=sharing}{\textcolor {orange} {\textbf{Stanford Autonomous Helicopter}} }).
%\end{frame}
%
%\begin{frame}{Reinforcement learning (2/3)}
%	\begin{center}
%		\includegraphics[scale=.325]{reinforcement-learning-2}
%	\end{center}
%	Use a "reward" signal to tell the AI when it is doing well or poorly. It automatically learns to maximize its rewards.
%	(Contoh: \href{https://drive.google.com/file/d/1Bia4pGf2m1lg9Fd8Suu732Hz471nUg6j/view?usp=sharing}{\textcolor {orange} {\textbf{AlphaGo is beating Humanity}}})
%\end{frame}
%
%\begin{frame}{Reinforcement learning (3/3)}
%	Dapatkah robot belajar bergerak di alam bebas? \\
%	Dengan implementasi yang tepat, \textit{it can be a walk in the park} \citep{smith2022awalk}.
%	
%	\bigskip
%	Link Video is \href{https://drive.google.com/file/d/1UE8JdqxJxpUn9McbEbfnRQXDLV9Pkmwl/view?usp=sharing}{\textcolor{orange}{\textbf{here}}}
%\end{frame}

\begin{frame}{AI Computing Platform}
	\begin{figure}[!ht]
		\centering
		\includegraphics[scale=.18]{images/peta-topik-huawei-ai}
	\end{figure}				
	Link Video: \href{https://drive.google.com/file/d/1e94HC381qGUsyLEfsxpE5o4btTd59HWZ/view?usp=sharing}{\textcolor{orange}{\textbf{here}}} \\
	Huawei HCIA-AI certification : \st{\$200} \textbf{\$0}.	
\end{frame}

\section{Future of AI: Generative AI}
\begin{frame}{Kebangkitan Generative AI}
	\begin{columns}
		\begin{column}{.5\textwidth}
			\begin{center}
				\includegraphics[width=\textwidth]{images/rise-generative-reuters.png}
			\end{center}
		\end{column}
		\begin{column}{.5\textwidth}
			Generative AI dapat
			\begin{itemize}
				\item<2-> Menambah \$2,6-\$4,4 triliun per tahun ke bidang ekonomi \citep{mckinsey2023economic}.
				\item<3-> Meningkatkan global GDP sebesar 7\% pada 10 tahun ke depan \citep{goldman2023large}.
				\item<4-> Berdampak 10\% dari tasks yang dikerjakan harian oleh 80\% dari pegawai \citep{eloundou2023gpts}.
			\end{itemize}			
		\end{column}
	\end{columns}
\end{frame}

\begin{frame}{Apakah Generative AI? (1/2)}
	Sistem Artificial Intelligence yang dapat menghasilkan high quality content, terutama untuk \textbf{text}, gambar (\textbf{images}), dan suara (\textbf{audio}). 
	
	\bigskip
	Beberapa contoh Generative AI adalah
			\begin{itemize}
	\item<2-> ChatGPT dari OpenAI
	\item<3-> Bard dari Google
	\item<4-> Bing Chat dari Microsoft
	\end{itemize}			
	
	\bigskip
	\onslide<5->Prompt $\Rightarrow$ pertanyaan yang meng-trigger Generative AI untuk me-respons.
\end{frame}

\begin{frame}{Apakah Generative AI? (2/2)}
	Generative AI juga merupakan \textbf{developer tool} \citep{ng2023generative}.
	\begin{center}
		\includegraphics[scale=.25]{images/generative-developer-tool.png}
	\end{center}
\end{frame}

\begin{frame}{Image, Audio, and Video Generation}
	\begin{columns}
		\begin{column}{.5\textwidth}
			A beautiful, pastoral mountain scene. Landscape painting style (\textbf{Midjourney})
			\begin{center}
				\includegraphics<2->[scale=.25]{images/contoh-midjourney.png}
			\end{center}
		\end{column}
		\begin{column}{.5\textwidth}
	\onslide<3->Two cute kittens playing (\textbf{DALL-E})
	\begin{center}
		\includegraphics<4->[scale=.25]{images/contoh-dall-e.png}
	\end{center}
\end{column}		
	\end{columns}
\end{frame}	

\begin{frame}{Menghasilkan Teks dengan Large Language Models (LLMs)}
	Text generation process \citep{ng2023generative}
	
	\begin{center}
		\includegraphics<2->[scale=.25]{images/prompt-and-llm.png}
	\end{center}
\end{frame}

\begin{frame}{Bagaimana LLMs belajar}
	LLMs dibuat dengan menggunakan supervised learning ($A \rightarrow B$) untuk memprediksi kata berikut secara berulang kali.
	
	\begin{center}
		\includegraphics<2->[scale=.25]{images/tabel-next-word.png}
	\end{center}
	
	\onslide<3->{Ketika kita melatih sistem AI yang sangat besar pada a lot of data (ratusan miliar kata), kita memperoleh Large Language Model  \\ seperti ChatGPT.}
\end{frame}

\begin{frame}{Cara Baru Mencari Informasi}
	\begin{columns}
		\begin{column}{.5\textwidth}
			\includegraphics<2->[scale=.2]{images/cari-informasi-1.png}
		\end{column}
		\begin{column}{.5\textwidth}
			\includegraphics<3->[scale=.2]{images/cari-informasi-2.png}
		\end{column}		
	\end{columns}
\end{frame}

\begin{frame}{Partner Menulis \citep{ng2023generative}}
	\begin{columns}
	\begin{column}{.5\textwidth} 
		\includegraphics<2->[scale=.2]{images/partner-writing-1.png}
	\end{column}
	\begin{column}{.5\textwidth}
		\includegraphics<3->[scale=.2]{images/partner-writing-2.png}
	\end{column}		
\end{columns}	
\end{frame}

\begin{frame}{Web search atau using an LLM \citep{ng2023generative}}
	\begin{columns}
		\begin{column}{.5\textwidth} 
			\begin{center}
				\includegraphics<2->[scale=.2]{images/web-search.png}
			\end{center}			
		\end{column}
		\begin{column}{.5\textwidth}
			\begin{center}
				\includegraphics<3->[scale=.2]{images/llm.png}
			\end{center}			
		\end{column}		
	\end{columns}	
\end{frame}

\begin{frame}{Web search atau using an LLM \citep{ng2023generative}}
	\begin{columns}
		\begin{column}{.5\textwidth} 
			\begin{center}
				\includegraphics<2->[scale=.2]{images/recipe.png}
			\end{center}			
		\end{column}
		\begin{column}{.5\textwidth}
			\begin{center}
				\includegraphics<3->[scale=.2]{images/recipe-2.png}
			\end{center}			
		\end{column}		
	\end{columns}	
\end{frame}

\begin{frame}{Web search atau using an LLM \citep{ng2023generative}}
	\begin{columns}
		\begin{column}{.5\textwidth} 
			\begin{center}
				\includegraphics<2->[scale=.2]{images/coffee-1.png}
			\end{center}			
		\end{column}
		\begin{column}{.5\textwidth}
			\begin{center}
				\includegraphics<3->[scale=.2]{images/coffee-2.png}
			\end{center}			
		\end{column}		
	\end{columns}	
\end{frame}

\begin{frame}{Contoh Tasks yang LLMs dapat Lakukan}
	\begin{columns}
		\begin{column}{.33\textwidth} 
			\begin{center}
				\includegraphics<2->[scale=.2]{images/writing.png}
			\end{center}			
		\end{column}
		\begin{column}{.33\textwidth}
			\begin{center}
				\includegraphics<3->[scale=.2]{images/reading.png}
			\end{center}			
		\end{column}		
		\begin{column}{.33\textwidth}
	\begin{center}
		\includegraphics<4->[scale=.2]{images/chatting.png}
	\end{center}			
\end{column}				
	\end{columns}	
\end{frame}

\begin{frame}{Contoh Tasks yang LLMs dapat Lakukan}
	\begin{columns}
		\begin{column}{.33\textwidth} 
			\begin{center}
				\includegraphics<2->[scale=.2]{images/writing-2.png}
			\end{center}			
		\end{column}
		\begin{column}{.33\textwidth}
			\begin{center}
				\includegraphics<3->[scale=.2]{images/reading-2.png}
			\end{center}			
		\end{column}		
		\begin{column}{.33\textwidth}
			\begin{center}
				\includegraphics<4->[scale=.2]{images/chatting-2.png}
			\end{center}			
		\end{column}				
	\end{columns}	
\end{frame}

\begin{frame}{Web-based vs. Software application use of LLMs}
	\begin{columns}
		\begin{column}{.5\textwidth} 
			\begin{center}
				\includegraphics<2->[scale=.2]{images/web-based.png}
			\end{center}			
		\end{column}
		\begin{column}{.5\textwidth}
			\begin{center}
				\includegraphics<3->[scale=.2]{images/software-based.png}
			\end{center}			
		\end{column}		
	\end{columns}	
\end{frame}

\begin{frame}{Demo: Mencoba Kode Generative AI}
	\begin{enumerate}
		\item<2-> Sentiment Analysis dari Restaurant Review.
		
		\bigskip
		\item<3-> Sistem Reputation Monitoring. 
	\end{enumerate}
\end{frame}

\begin{frame}{Writing: Brainstorming product names}
	\begin{center}
		\includegraphics<2->[scale=.25]{images/brainstorm}
	\end{center}
\end{frame}

\begin{frame}{Writing: Developing sales strategy}
	\begin{center}
		\includegraphics<2->[scale=.25]{images/develop-strategy}
	\end{center}
\end{frame}

\begin{frame}{Writing: Memberikan ide riset terbaru}
	\begin{center}
		\includegraphics<2->[scale=.45]{images/ide-riset}
	\end{center}
\end{frame}

\begin{frame}{Reading: Proofreading}
	\begin{center}
		\includegraphics<2->[scale=.225]{images/proofread}
	\end{center}
\end{frame}

\begin{frame}{Reading: Merangkum Artikel Ilmiah}
	\begin{center}
		\includegraphics<2->[scale=.225]{images/rangkuman}
	\end{center}
\end{frame}

\begin{frame}{Reading: Reputation Monitoring}
	\begin{center}
		\includegraphics<2->[scale=.175]{images/reputation-monitoring}
	\end{center}
\end{frame}

\begin{frame}{Chatting: ChatBot Khusus}
	\begin{center}
		\includegraphics<2->[scale=.2]{images/chatbot-khusus}
	\end{center}
\end{frame}

\begin{frame}{Chatting: IT Service ChatBot}
	\begin{center}
		\includegraphics<2->[scale=.25]{images/it-chatbot}
	\end{center}
\end{frame}

\begin{frame}{Chatting: The Rise of Chatbots in Customer Service}
	\begin{center}
		\includegraphics<2->[scale=.18]{images/rise-chatbots}
	\end{center}
\end{frame}

\begin{frame}{Kesimpulan}
	\begin{enumerate}
		\item<2-> AI dapat dibagi menjadi 2, yaitu: ANI dan AGI.
		\item<3-> \textit{Supervised Learning} merupakan tipe algoritma machine learning yang umum digunakan saat ini. 
		\item<4-> \textit{Supervised learning} sangat membutuhkan dataset yang mempunyai A (input) dan B (output).
		\item<5-> \textit{Artificial Neural Networks} merupakan fondasi dari teknologi \textbf{Deep Learning}.
		\item<6-> \textit{Feature extraction} secara otomatis merupakan kekuatan dari Deep Learning.
		\item<7-> \textbf{Large Language Models} merupakan fondasi dari Generative AI $\Rightarrow$ Future AI.
		\item<8-> LLMs dapat diaplikasikan dalam banyak aplikasi, contohnya: writing, reading, chatting, dan masih banyak peluang lainnya.
	\end{enumerate}
\end{frame}





%\begin{frame}{GANs (Generative Adversarial Network) (1/3)}
%	\begin{center}
%		Synthesize new images from scratch~\citep{karras2017progressive} (\href{https://drive.google.com/file/d/12kyeMlwG7geHo3WzO9wOo00n53LBOvpj/view?usp=sharing}{\textcolor{orange}{\textbf{GANs in Action}}})
%	\end{center}
%	\begin{center}
%		\includegraphics[scale=.25]{GANs}
%	\end{center}			
%\end{frame}

%\begin{frame}{GANs (Generative Adversarial Network) (2/3)}
%	\textbf{GauGan} won a 2019 PopSci Magazine "Best of What's New Award" in the engineering category.
%	
%	\bigskip
%	Developed by NVIDIA researchers in 2019, GauGAN is the first AI model that can produce complex images with only a few brushstrokes. 
%	
%	\bigskip
%	Video Link is \href{https://twitter.com/NVIDIAAIDev/status/1201907407242153986}{\textcolor{orange}{\textbf{here}} }.
%	
%	\bigskip
%	Aplikasi yang mirip: \href{https://affinelayer.com/pixsrv/?utm_source=pocket_reader}{\textcolor{orange}{\textbf{Image-to-Image Demo}}}	
%\end{frame}

%\begin{frame}{GANs (Generative Adversarial Network) (3/3)}
%	\begin{itemize}
%		\item Using AI to generate fashion by utilizing DALL-E. \\
%		Video Link is \href{https://drive.google.com/file/d/1xrvBm1TKbGASN8CApdZkgL8sP5aXW2T_/view?usp=sharing}{\textcolor{orange}{\textbf{here}} }.
%		
%		\bigskip
%		\item DALL-E: Introducing Outpainting \\
%		Extend creativity and tell a bigger story with DALL-E images of any size. \\
%		Video Link is \href{https://drive.google.com/file/d/1mJzmS_tayjQWHqwH9WyqvOdv9NGkOBMs/view?usp=share_link}{\textcolor{orange}{\textbf{here}} }.		
%	\end{itemize}	
%\end{frame}





%\noindent\fbox{%
%    \parbox{\textwidth}{%
%        The quick brown fox jumps right over the lazy dog. 
%    }%
%}











%\begin{frame}{Blocks}
%\begin{block}{Block Title}
%You can also highlight sections of your presentation in a block, with it's own title
%\end{block}
%\begin{theorem}
%There are separate environments for theorems, examples, definitions and proofs.
%\end{theorem}
%\begin{example}
%Here is an example of an example block.
%\end{example}
%\end{frame}


% All of the following is optional and typically not needed. 
\appendix
\section<presentation>*{\appendixname}
\subsection<presentation>*{For Further Reading}

\begin{frame}[allowframebreaks]
  \frametitle<presentation>{Daftar Pustaka}
    {\footnotesize
    \bibliographystyle{apalike}
    \bibliography{references}
    }    
\end{frame}




%\makeatletter % to change template
%    \setbeamertemplate{headline}[default] % not mandatory, but I though it was better to set it blank
%    \def\beamer@entrycode{\vspace*{-\headheight}} % here is the part we are interested in :)
%\makeatother

\begin{frame}[plain]
		\centering\includegraphics[scale=0.5]{Logo-Maranatha-Untuk-Belakang-02}	
\end{frame}

\end{document}



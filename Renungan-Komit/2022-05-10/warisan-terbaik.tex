\documentclass[10pt,svgnames]{beamer} %Beamer
\usepackage{palatino} %font type
\usefonttheme{metropolis} %Type of slides
\usefonttheme[onlymath]{serif} %font type Mathematical expressions
\usetheme[progressbar=frametitle,titleformat frame=smallcaps,numbering=counter]{metropolis} %This adds a bar at the beginning of each section.
\useoutertheme[subsection=false]{miniframes} %Circles in the top of each frame, showing the slide of each section you are at

\usepackage{appendixnumberbeamer} %enumerate each slide without counting the appendix
\setbeamercolor{progress bar}{fg=Maroon!70!Coral} %These are the colours of the progress bar. Notice that the names used are the svgnames
\setbeamercolor{title separator}{fg=DarkSalmon} %This is the line colour in the title slide
\setbeamercolor{structure}{fg=black} %Colour of the text of structure, numbers, items, blah. Not the big text.
\setbeamercolor{normal text}{fg=black!87} %Colour of normal text
\setbeamercolor{alerted text}{fg=DarkRed!60!Gainsboro} %Color of the alert box
\setbeamercolor{example text}{fg=Maroon!70!Coral} %Colour of the Example block text


\setbeamercolor{palette primary}{bg=NavyBlue!50!DarkOliveGreen, fg=white} %These are the colours of the background. Being this the main combination and so one. 
\setbeamercolor{palette secondary}{bg=NavyBlue!50!DarkOliveGreen, fg=white}
\setbeamercolor{palette tertiary}{bg=NavyBlue!40!Black, fg= white}
\setbeamercolor{section in toc}{fg=NavyBlue!40!Black} %Color of the text in the table of contents (toc)

% =========================
%   Added Packages
% =========================
\usepackage{natbib}

%These next packages are the useful for Physics in general, you can add the extras here. 
\usepackage{amsmath,amssymb}
\usepackage{slashed}
\usepackage{cite}
\usepackage{relsize}
\usepackage{caption}
\usepackage{subcaption}
\usepackage{multicol}
\usepackage{booktabs}
\usepackage[scale=2]{ccicons}
\usepackage{pgfplots}
\usepgfplotslibrary{dateplot}
\usepackage{geometry}
\usepackage{xspace}
\newcommand{\themename}{\textbf{\textsc{bluetemp}\xspace}}%metropolis}}\xspace}

\title{The Best Legacy (Warisan yang Terbaik)}
\author[Name]{Ev. Veronica} %With inst, you can change the institution they belong
%\subtitle{Rangkuman dari Social Signal Processing}
%\institute[uni]{\inst{$\dagger$} Department of blah blah \\ University of \LaTeX}
\date{\today} %Here you can change the date
%\titlegraphic{\vspace{-0.5cm}\hfill\includegraphics[scale=0.23]{images/logo}} %You can modify the location of the logo by changing the command \vspace{}. 

\begin{document}
{
\setbeamercolor{background canvas}{bg=NavyBlue!50!DarkOliveGreen, fg=white}
\setbeamercolor{normal text}{fg=white}
\maketitle
}%This is the colour of the first slide. bg= background and fg=foreground

\metroset{titleformat frame=smallcaps} %This changes the titles for small caps

\begin{frame}{Outline}
  \setbeamertemplate{section in toc}[sections numbered] %This is numbering the sections
  \tableofcontents[hideallsubsections] %You can comment this line if you want to show the subsections in the table of contents
\end{frame}

\section{Ayat Firman Tuhan}
\begin{frame}{Firman Tuhan: Titus 2:3-5 (1/2)}
	Berkaitan dengan peringatan Hari Ibu Internasional, Firman Tuhan terambil dari \textbf{Titus 2:3-5}.
	
	\bigskip
	Mari kita baca bersama-sama.
\end{frame}

\begin{frame}{Firman Tuhan: Titus 2:3-5 (2/2)}
	\begin{block}{Titus 2:3-5}
		Demikian juga perempuan-perempuan yang tua, hendaklah mereka hidup sebagai \onslide<2->{\textbf{orang-orang beribadah}}, \onslide<3->{\textbf{jangan memfitnah}}, \onslide<4->{\textbf{jangan menjadi hamba anggur}}, tetapi \onslide<5->{\textbf{cakap mengajarkan hal-hal yang baik}}  dan dengan demikian \onslide<6->{\textbf{mendidik perempuan-perempuan muda mengasihi suami dan anak-anaknya}}, \onslide<7->{\textbf{hidup bijaksana dan suci}}, \onslide<8->{\textbf{rajin mengatur rumah tangganya}}, \onslide<9->{\textbf{baik hati dan taat kepada suaminya}}, agar Firman Allah jangan dihujat orang.
	\end{block}	
\end{frame}


\begin{frame}{Nasihat untuk Para Wanita (Dewasa)}
	\begin{itemize}
		\item<2-> Hidup sebagai orang yang beribadah 
		\item<3-> Hidup dengan menunjukkan sikap hormat kepada Allah melalui perilaku
		\item<4-> Tidak hidup dalam dosa
		\item<5-> Tidak memfitnah dan tidak menjadi hamba anggur
		\item<6-> Cakap mengajar		
	\end{itemize}
\end{frame}

\begin{frame}{Tujuan Akhir}
	Wanita yang 
	\begin{itemize}
		\item<2-> mengasihi suami dan anak
		\item<3-> mengurus keluarga dengan baik
		\item<4-> tidak menjadi batu sandungan
	\end{itemize}
    \onslide<5->{Membesarkan anak bukan hal yang mudah termasuk juga bagi para Hamba Tuhan.} \\
    \onslide<6->{Masing-masing mempunyai pergumulannya sendiri-sendiri karena anak-anak bukan robot.}  \\ 
    \onslide<7->{Anak-anak yang dibesarkan di lingkungan yang baik belum tentu otomatis bertumbuh menjadi anak yang baik. \\ 
    	Contoh: keluarga Imam Eli dan Nabi Samuel.}    
\end{frame}

\section{Empat Tips untuk Mewariskan Iman}
\begin{frame}{Empat Tips Mewariskan Iman kepada Anak-anak}
	\begin{enumerate}
		\item<2-> \textbf{Teladan}
		\item<3-> \textbf{Sedini mungkin}
		\item<4-> \textbf{Ajarkan Firman Tuhan}
		\item<5-> \textbf{Kabarkan dan Kobarkan} 
	\end{enumerate}
\end{frame}

\begin{frame}{Tips \#1: Teladan}
	\begin{itemize}
		\item<2-> Menjadi pribadi yang terbuka dan utuh agar terlihat oleh orang lain (2 Timotius 1:5).\\
		 \onslide<3->{"\textit{Sebab aku teringat akan imanmu yang tulus ikhlas, yaitu iman yang pertama-tama hidup di dalam nenekmu Lois dan di dalam ibumu Eunike dan yang aku yakin hidup juga di dalam dirimu.}".}
		\item<4-> Kita juga melihat hal yang sama ketika Tuhan Yesus memberikan keteladanan hidup-Nya kepada murid-murid-Nya \onslide<5->{$\Longrightarrow$ suatu proses yang berlanjut terus.} 
		\item<6-> Dalam kehidupan kita pun, anak-anak akan meneladani bagaimana kita bertindak dan bertingkah laku.		
	\end{itemize}	    
\end{frame}

\begin{frame}{Tips \#2: Sedini Mungkin (1/2)}
	\begin{itemize}
		\item<2-> Hal ini mengingatkan kita kepada \textbf{shema Yisrael} yang diajarkan oleh setiap orang tua Israel kepada anak-anak mereka \citep{shemayisrael}.
		\item<3-> Shema Yisrael dipandang sebagai doa yang paling penting di dalam agama Yahudi dan penyebutannya dua kali dalam sehari adalah sebuah \textit{mitzvah} (perintah rohani).
		\item<4-> Pada awalnya, Shema hanya terdiri dari satu ayat: \textbf{Ulangan 6:4}.
		\item<5-> Penceritaan Shema di dalam upacara doa namun terdiri dari tiga bagian: \textbf{Ulangan 6:4-9}, \textbf{11:13-21}, dan \textbf{Bilangan 15:37-41}.  
	\end{itemize}
\end{frame}

\begin{frame}{Tips \#2: Sedini Mungkin (2/2)}
	\begin{itemize}
		\item<2-> Lois dan Eunike mengenalkan Kitab Suci di masa kecil Timotius. 
		\item<3-> Ayah dan Ibu Samuel mengajarkan taat kepada Tuhan di masa kecil Samuel.
		\item<4-> Demikian juga dengan Tuhan Yesus yang sudah dibawa oleh orang tua-Nya ketika berusia 8 hari ke Bait Suci dan saat Tuhan Yesus berusia 12 tahun dalam perayaan Paskah.
		\item<5->  Penting untuk mendisiplin anak saat mereka berbohong atau tidak mau ibadah ketimbang saat mereka memecahkan kacamata, piring, dll.
	\end{itemize}
\end{frame}

\begin{frame}{Tips \#3: Ajarkan Firman Tuhan}
	\begin{itemize}
		\item<2-> Firman Tuhan menjadi otoritas tertinggi yang mempengaruhi hidup kita. 
		\item<3-> Hanya dengan Firman Tuhan yang akan menuntun kita dalam melangkah dan membuat kita bisa bangkit kembali dari segala kegagalan.
	\end{itemize}
\end{frame}

\begin{frame}{Tips \#4: Kabarkan dan Kobarkan}
	 \onslide<2->{Kita tidak bisa mengabarkan Injil ke seluruh dunia, tetapi kita bisa mengabarkan Injil kepada anak-anak kita.} 
\end{frame}

\section{Pertanyaan Refleksi}
\begin{frame}{Aplikasi: Pertanyaan Refleksi}
	\begin{itemize}
		\item<2-> Bagaimanakah kabar dengan orang-orang terdekat yang telah membawa anda kepada Tuhan?
		\item<3-> Apakah ada di antara teman-teman yang mau share tentang hal ini?
	\end{itemize}		
\end{frame}

\appendix
\begin{frame}[allowframebreaks]
	\frametitle<presentation>{}
	{\footnotesize
		\bibliographystyle{apalike}
		\bibliography{references}
	}    
\end{frame}

%\begin{frame}{Back up}
%    These slides won't appear in the table of contents and will not be counted as the total slides.
%\end{frame}
{\setbeamercolor{palette primary}{fg=black, bg=white!30} %You can change the colours
	\begin{frame}[standout]
		%  Thank you! And thank to yourself because you did all the job. 
		\centering\includegraphics[scale=1]{images/thank-you}	
		hendra.bunyamin@it.maranatha.edu
	\end{frame}

\end{document}

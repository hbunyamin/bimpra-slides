\documentclass[10pt,svgnames]{beamer} %Beamer
\usepackage{palatino} %font type
\usefonttheme{metropolis} %Type of slides
\usefonttheme[onlymath]{serif} %font type Mathematical expressions
\usetheme[progressbar=frametitle,titleformat frame=smallcaps,numbering=counter]{metropolis} %This adds a bar at the beginning of each section.
\useoutertheme[subsection=false]{miniframes} %Circles in the top of each frame, showing the slide of each section you are at

\usepackage{appendixnumberbeamer} %enumerate each slide without counting the appendix
\setbeamercolor{progress bar}{fg=Maroon!70!Coral} %These are the colours of the progress bar. Notice that the names used are the svgnames
\setbeamercolor{title separator}{fg=DarkSalmon} %This is the line colour in the title slide
\setbeamercolor{structure}{fg=black} %Colour of the text of structure, numbers, items, blah. Not the big text.
\setbeamercolor{normal text}{fg=black!87} %Colour of normal text
\setbeamercolor{alerted text}{fg=DarkRed!60!Gainsboro} %Color of the alert box
\setbeamercolor{example text}{fg=Maroon!70!Coral} %Colour of the Example block text


\setbeamercolor{palette primary}{bg=NavyBlue!50!DarkOliveGreen, fg=white} %These are the colours of the background. Being this the main combination and so one. 
\setbeamercolor{palette secondary}{bg=NavyBlue!50!DarkOliveGreen, fg=white}
\setbeamercolor{palette tertiary}{bg=NavyBlue!40!Black, fg= white}
\setbeamercolor{section in toc}{fg=NavyBlue!40!Black} %Color of the text in the table of contents (toc)

% =========================
%   Added Packages
% =========================
\usepackage{natbib}

%These next packages are the useful for Physics in general, you can add the extras here. 
\usepackage{amsmath,amssymb}
\usepackage{slashed}
\usepackage{cite}
\usepackage{relsize}
\usepackage{caption}
\usepackage{subcaption}
\usepackage{multicol}
\usepackage{booktabs}
\usepackage[scale=2]{ccicons}
\usepackage{pgfplots}
\usepgfplotslibrary{dateplot}
\usepackage{geometry}
\usepackage{xspace}
\newcommand{\themename}{\textbf{\textsc{bluetemp}\xspace}}%metropolis}}\xspace}

\title{Murid Kristus yang Hidup dalam Kemerdekaan yang Sejati}
\author[Name]{Ev. Lily Efferin} %With inst, you can change the institution they belong
%\subtitle{Rangkuman dari Social Signal Processing}
%\institute[uni]{\inst{$\dagger$} Department of blah blah \\ University of \LaTeX}
\date{\today} %Here you can change the date
%\titlegraphic{\vspace{-0.5cm}\hfill\includegraphics[scale=0.23]{images/logo}} %You can modify the location of the logo by changing the command \vspace{}. 

\begin{document}
{
\setbeamercolor{background canvas}{bg=NavyBlue!50!DarkOliveGreen, fg=white}
\setbeamercolor{normal text}{fg=white}
\maketitle
}%This is the colour of the first slide. bg= background and fg=foreground

\metroset{titleformat frame=smallcaps} %This changes the titles for small caps

\begin{frame}{Outline}
  \setbeamertemplate{section in toc}[sections numbered] %This is numbering the sections
  \tableofcontents[hideallsubsections] %You can comment this line if you want to show the subsections in the table of contents
\end{frame}

\section{Ayat Firman Tuhan}
\begin{frame}{Firman Tuhan: Galatia 5:1 (1/3)}
Supaya kita sungguh-sungguh merdeka, \textbf{Kristus telah memerdekakan kita}. Karena itu \textbf{berdirilah teguh} dan \textbf{jangan mau lagi dikenakan kuk perhambaan}.	
\end{frame}

\begin{frame}{Firman Tuhan: Galatia 5:13-14 (2/3)}
Saudara-saudara, memang kamu telah dipanggil untuk merdeka. Tetapi janganlah kamu mempergunakan kemerdekaan itu sebagai kesempatan untuk kehidupan dalam dosa, melainkan layanilah seorang akan yang lain oleh kasih. Sebab seluruh hukum Taurat tercakup dalam satu firman ini, yaitu: "Kasihilah sesamamu manusia seperti dirimu sendiri!"
\end{frame}

\begin{frame}{Firman Tuhan: 1 Petrus 2:16 (3/3)}
	Hiduplah sebagai orang merdeka dan bukan seperti mereka yang menyalahgunakan kemerdekaan itu untuk menyelubungi kejahatan-kejahatan mereka, tetapi hiduplah sebagai hamba Allah.	
\end{frame}

\section{Renungan}
\begin{frame}{Kebebasan yang Sejati}
	\begin{itemize}
		\item Kebebasan yang sejati \textbf{bukan} kebebasan tanpa batas.
		\item Kebebasan bukan hak semata-mata, tetapi juga ada tanggung jawab.
	\end{itemize}	
\end{frame}

\begin{frame}{Kemerdekaan dalam Kristus}
	\begin{itemize}
		\item Kemerdekaan dalam Kristus tidak dapat terlepas dari \textbf{Karya Kristus}.
		\item Kemerdekaan sejati menurut Alkitab dalam karya Kristus, sebagai anugerah Allah bukan hak kita.
	\end{itemize}
\end{frame}

\begin{frame}{Dimerdekaan dari ...}
	Dimerdekaan dari (Efesus 2:1-3, Kolose 3:5-10)
	\begin{itemize}
		\item Manusia lama yang dikuasai oleh dosa.
		\item Dominasi Iblis dan segala tipu dayanya.
		\item Upah dosa (maut) dan intimidasinya.
	\end{itemize}
\end{frame}

\begin{frame}{Dimerdekaan untuk ...}
	Dimerdekaan untuk (1 Petrus 2:9,16)
	\begin{itemize}
		\item Hidup sebagai anak Allah.
		\item Hidup dalam kemenangan.
		\item Bersaksi dan melayani dalam kasih.
	\end{itemize}
	Hiduplah sebagai orang yang sudah dimerdekakan.
\end{frame}


\begin{frame}{Tetap Hidup dalam Kemerdekaan}
	\begin{itemize}
		\item Percaya dan berdiri teguh.
		\item Jangan mau lagi dikenakan kuk perhambaan.
		\item Jangan salah menggunakan kemerdekaan.
		\item Hidup dalan komunitas yang melayani dengan kasih.
	\end{itemize}
\end{frame}

\section{Pertanyaan Refleksi}
\begin{frame}{Refleksi}
	\begin{enumerate}
		\item<2-> Apakah Anda sudah memiliki kemerdekaan yang Tuhan Yesus (Alkitab) maksudkan? Bagaimana Anda memastikannya?
		\item<3-> Menurut Anda, sebesar apakah pengorbanan Kristus untuk menebus manusia berdosa (Anda \& saya)? Sedalam apa pemahamanan dan penghargaan Anda akan pengorbanan Kristus demi memberikan kemerdekaan sejati sedemikian?
		\item<4-> Secara spesifik, kemerdekaan dalam aspek apa saja yang Kristus bisa dan ingin wujudkan dalam hidup Anda? Percaya dan sudahkah Anda menerimanya? 
		\item<5-> Apa yang perlu dilakukan agar kemerdekaan yang demikian Kristus genapkan dalam hidup Anda?
		\item<6-> Apa yang bisa (perlu) dilakukan orang percaya agar tetap hidup dalam kemerdekaan Kristus?
	\end{enumerate}
\end{frame}


%\begin{frame}{Back up}
%    These slides won't appear in the table of contents and will not be counted as the total slides.
%\end{frame}
{\setbeamercolor{palette primary}{fg=black, bg=white!30} %You can change the colours
	\begin{frame}[standout]
		%  Thank you! And thank to yourself because you did all the job. 
		\centering\includegraphics[scale=1]{images/thank-you}	
		hendra.bunyamin@it.maranatha.edu
	\end{frame}

\end{document}

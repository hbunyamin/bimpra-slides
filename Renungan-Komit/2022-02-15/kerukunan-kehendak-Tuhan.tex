\documentclass[10pt,svgnames]{beamer} %Beamer
\usepackage{palatino} %font type
\usefonttheme{metropolis} %Type of slides
\usefonttheme[onlymath]{serif} %font type Mathematical expressions
\usetheme[progressbar=frametitle,titleformat frame=smallcaps,numbering=counter]{metropolis} %This adds a bar at the beginning of each section.
\useoutertheme[subsection=false]{miniframes} %Circles in the top of each frame, showing the slide of each section you are at

\usepackage{appendixnumberbeamer} %enumerate each slide without counting the appendix
\setbeamercolor{progress bar}{fg=Maroon!70!Coral} %These are the colours of the progress bar. Notice that the names used are the svgnames
\setbeamercolor{title separator}{fg=DarkSalmon} %This is the line colour in the title slide
\setbeamercolor{structure}{fg=black} %Colour of the text of structure, numbers, items, blah. Not the big text.
\setbeamercolor{normal text}{fg=black!87} %Colour of normal text
\setbeamercolor{alerted text}{fg=DarkRed!60!Gainsboro} %Color of the alert box
\setbeamercolor{example text}{fg=Maroon!70!Coral} %Colour of the Example block text


\setbeamercolor{palette primary}{bg=NavyBlue!50!DarkOliveGreen, fg=white} %These are the colours of the background. Being this the main combination and so one. 
\setbeamercolor{palette secondary}{bg=NavyBlue!50!DarkOliveGreen, fg=white}
\setbeamercolor{palette tertiary}{bg=NavyBlue!40!Black, fg= white}
\setbeamercolor{section in toc}{fg=NavyBlue!40!Black} %Color of the text in the table of contents (toc)

% =========================
%   Added Packages
% =========================
\usepackage{natbib}

%These next packages are the useful for Physics in general, you can add the extras here. 
\usepackage{amsmath,amssymb}
\usepackage{slashed}
\usepackage{cite}
\usepackage{relsize}
\usepackage{caption}
\usepackage{subcaption}
\usepackage{multicol}
\usepackage{booktabs}
\usepackage[scale=2]{ccicons}
\usepackage{pgfplots}
\usepgfplotslibrary{dateplot}
\usepackage{geometry}
\usepackage{xspace}
\newcommand{\themename}{\textbf{\textsc{bluetemp}\xspace}}%metropolis}}\xspace}

\title{Kerukunan = Berkat yang Tuhan Inginkan}
\author[Name]{Hendra Bunyamin} %With inst, you can change the institution they belong
%\subtitle{Rangkuman dari Social Signal Processing}
%\institute[uni]{\inst{$\dagger$} Department of blah blah \\ University of \LaTeX}
\date{\today} %Here you can change the date
%\titlegraphic{\vspace{-0.5cm}\hfill\includegraphics[scale=0.23]{images/logo}} %You can modify the location of the logo by changing the command \vspace{}. 

\begin{document}
{
\setbeamercolor{background canvas}{bg=NavyBlue!50!DarkOliveGreen, fg=white}
\setbeamercolor{normal text}{fg=white}
\maketitle
}%This is the colour of the first slide. bg= background and fg=foreground

\metroset{titleformat frame=smallcaps} %This changes the titles for small caps

\begin{frame}{Outline}
  \setbeamertemplate{section in toc}[sections numbered] %This is numbering the sections
  \tableofcontents[hideallsubsections] %You can comment this line if you want to show the subsections in the table of contents
\end{frame}

\section{Ayat Firman Tuhan}
\begin{frame}{Firman Tuhan: Mazmur 133:1-3}
	Mari kita baca bersama-sama.
\end{frame}

\section{Kupasan Firman Tuhan}
\begin{frame}{Kerukunan adalah Kehendak Tuhan}
	\begin{itemize}
		\item<2-> Kerukunan adalah kehendak Tuhan yang \textbf{menyenangkan} dan \textbf{memberikan sukacita}.
		\item<3-> Kerukunan adalah \textbf{kebaikan}.
	\end{itemize}
	\onslide<4->{Coba bandingkan dengan \textbf{didikan} di kitab Ibrani 12:11 $\longrightarrow$ tidak menyenangkan}
\end{frame}

\begin{frame}{Kerukunan seperti \textbf{Minyak}}
	\begin{itemize}
		\item<2-> Minyak memberikan \textit{kesegaran} dan \textit{kelembaban} dalam cuaca yang panas \& kering.
		\item<3-> Minyak yang \textit{meleleh} dari kepala ke janggut $\longrightarrow$ kelimpahan berkat Tuhan.
		\item<4-> Minyak yang \textit{meleleh} dari janggut ke dada $\longrightarrow$ kerukunan 12 suku (antara saudara seiman).
	\end{itemize}
\begin{center}
	\includegraphics<5->[scale=.2]{images/baju-harun-detil}
\end{center}
\end{frame}

\begin{frame}{Peta Gunung Hebron \& Gunung-Gunung Sion (1/2)}
	\begin{center}
		\includegraphics<2->[scale=.4]{images/hebron-map}
	\end{center}
\end{frame}

\begin{frame}{Peta Gunung Hebron \& Gunung-Gunung Sion (2/2)}
	\begin{center}
		\includegraphics<2->[scale=.25]{images/gunung-sion}
	\end{center}
\end{frame}

\begin{frame}{Kerukunan seperti \textbf{Embun}}
	\begin{itemize}
		\item<2-> Embun dari Gunung Hebron ($\pm 2.750$ km) sampai ke Gunung Zion sejauh $\pm 200$ km. 
		\item<3-> Embun $\longrightarrow$ berkat, kehidupan untuk selama-lamanya $\longrightarrow$ icip-icip kehidupan surgawi.
	\end{itemize}
\end{frame}

\section{Dasar \& Dampak Hidup Rukun}
\begin{frame}{Dasar Hidup Rukun}
	\begin{itemize}
		\item<2-> \textbf{Unsur Iman Percaya} \\
		\onslide<3->{Selalu ingat bahwa saya \textbf{orang berdosa} yang sudah diselamatkan $\longrightarrow$ kerendahan hati.} 		
		\item<4-> \textbf{Unsur Relasi Kasih} \\
		\onslide<5->{\textbf{Yohanes 13:33-35} $\longrightarrow$ Perintah Baru.}\\
		\onslide<6->{Kasih yang ditunjukkan oleh Tuhan Yesus.}
		\item<7-> \textbf{Unsur Disiplin Rohani} \\
		\onslide<8-> Kedisiplinan dalam merenungkan Firman Tuhan dan juga ber-KOMIT-ria.
 	\end{itemize}	
\end{frame}

\begin{frame}{Dampak Hidup Rukun}
	\begin{itemize}
	\item<2-> Menghadirkan sampel kehidupan surgawi.
	\item<3-> Menjadikan kasih pengorbanan Tuhan Yesus nyata.
	\item<4-> Kesaksian hidup yang berkuasa (Kis 2:47)
\end{itemize}		
\end{frame}

\begin{frame}{Aplikasi: Pertanyaan Refleksi}
	\begin{itemize}
		\item<2-> Apakah ada di antara teman-teman yang masih belum rukun dengan sesamanya?
		\item<3-> Apabila ada, bagaimanakah langkah teman-teman untuk menghadirkan kerukunan mengingat bahwa kerukunan membawa berkat bagi teman-teman dan juga sudah diamanatkan oleh Tuhan?
	\end{itemize}		
\end{frame}

\appendix
%\begin{frame}[allowframebreaks]
%	\frametitle<presentation>{}
%	{\footnotesize
%		\bibliographystyle{apalike}
%		\bibliography{references}
%	}    
%\end{frame}

%\begin{frame}{Back up}
%    These slides won't appear in the table of contents and will not be counted as the total slides.
%\end{frame}
{\setbeamercolor{palette primary}{fg=black, bg=white!30} %You can change the colours
	\begin{frame}[standout]
		%  Thank you! And thank to yourself because you did all the job. 
		\centering\includegraphics[scale=1]{images/thank-you}	
		hendra.bunyamin@it.maranatha.edu
	\end{frame}

\end{document}
